\documentclass[12pt]{article}
\usepackage[utf8]{inputenc}
\usepackage{graphicx}
\usepackage{amsmath}
\usepackage{amssymb}
\usepackage{hyperref}
\usepackage[margin=1in]{geometry}
\usepackage{booktabs}
\usepackage{longtable}

\title{Classifying Awakened Abilities: A Typology and Research Agenda}
\author{Taro Nakamura\textsuperscript{1} \and Sakura Yamamoto\textsuperscript{2} \\
\small \textsuperscript{1}Department of Supernatural Studies, University of Tokyo, Japan\\
\small \textsuperscript{2}Research Institute for Awakened Phenomena, Kyoto, Japan}
\date{August 2026}

\begin{document}

\maketitle

\begin{abstract}
In the aftermath of the 2025 catastrophe, which saw the emergence of portals leading to perilous dungeons and the awakening of a small portion of the population with extraordinary abilities, the need for a comprehensive typology of awakened abilities has become increasingly pressing. This paper proposes a classification system based on the Fujita-Sato (F-S) scale, which categorizes abilities according to their power level and potential impact. We review existing research on the nature and distribution of awakened abilities, identify key research gaps, and outline a research agenda for advancing our understanding of this phenomenon. Our findings highlight the importance of interdisciplinary collaboration, longitudinal studies, and the integration of awakened individuals' perspectives in future research efforts.
\end{abstract}

\section{Introduction}
The 2025 catastrophe marked a turning point in human history, as the sudden appearance of portals leading to treacherous, monster-infested dungeons was accompanied by the awakening of a small portion of the population with extraordinary abilities \cite{Kim2027}. These "awakened" individuals quickly became the focus of intense scientific, political, and public interest, as their abilities were seen as both a potential resource and a threat in the post-catastrophe world \cite{Chen2026, Ishikawa2026}.

Despite the growing body of research on awakened abilities, there remains a lack of consensus on how to classify and understand the diverse range of powers exhibited by these individuals. Existing classification systems, such as the Yamada-Watanabe scale \cite{Yamada2025} and the Kang-Park index \cite{Kang2026}, have been criticized for their limited scope and reliance on subjective criteria. As a result, there is a pressing need for a more comprehensive and rigorous typology of awakened abilities, which can serve as a foundation for future research and policy-making.

In this paper, we propose a new classification system based on the Fujita-Sato (F-S) scale, which categorizes abilities according to their power level and potential impact. The F-S scale ranges from F (low power, localized impact) to S (high power, global impact), with intermediate levels denoted by letters E through A. We argue that this scale provides a more nuanced and objective framework for understanding the diversity of awakened abilities, while also accounting for the complex interplay between individual powers and the broader social, economic, and political context of the post-catastrophe world.

To develop our typology, we draw upon a comprehensive review of existing research on awakened abilities, as well as original data from a nationwide survey of awakened individuals in Japan (n=3,000) and in-depth interviews with a subset of respondents (n=100). Our analysis reveals several key dimensions along which awakened abilities can be classified, including the type of ability (e.g., physical, mental, elemental, magical), the scope of impact (e.g., personal, interpersonal, societal), and the degree of control and mastery exhibited by the individual.

Building on these findings, we propose a research agenda for advancing our understanding of awakened abilities and their implications for the post-catastrophe world. We argue that future research should prioritize interdisciplinary collaboration, drawing on insights from fields such as biology, psychology, sociology, and political science. We also highlight the need for longitudinal studies that can track the development and evolution of awakened abilities over time, as well as comparative studies that can shed light on the cultural and institutional factors shaping the expression and regulation of these powers in different contexts.

Finally, we emphasize the importance of integrating the perspectives and experiences of awakened individuals themselves into future research efforts. While much of the existing literature on awakened abilities has focused on the perspectives of researchers, policy-makers, and the general public, there is a growing recognition of the need to center the voices and agency of those who possess these extraordinary powers. By engaging awakened individuals as active participants and co-creators of knowledge, we argue that researchers can develop a more nuanced and empowering understanding of this phenomenon.

\section{The Fujita-Sato (F-S) Scale}
The Fujita-Sato (F-S) scale is a classification system for awakened abilities that was developed by researchers at the University of Tokyo and the Research Institute for Awakened Phenomena in Kyoto \cite{Fujita2026}. The scale is based on two key dimensions: power level and potential impact.

Power level refers to the raw strength and magnitude of an awakened individual's abilities, as measured by a combination of objective tests and expert assessments. The F-S scale distinguishes between six power levels, ranging from F (low power) to S (high power), with intermediate levels denoted by letters E through A.

Potential impact, on the other hand, refers to the broader social, economic, and political consequences of an awakened individual's abilities, taking into account factors such as the scope of impact (e.g., personal, interpersonal, societal), the degree of control and mastery exhibited by the individual, and the potential for unintended or secondary effects. The F-S scale distinguishes between four levels of potential impact: localized (L), regional (R), national (N), and global (G).

To determine an individual's F-S classification, researchers and assessors first evaluate their power level using a battery of tests and measures, such as physical strength tests, mental aptitude tests, and simulations of ability usage in controlled environments. These evaluations are then combined with expert judgments and contextual information to arrive at a holistic assessment of the individual's power level.

Next, researchers and assessors consider the potential impact of the individual's abilities, taking into account factors such as the nature of the ability (e.g., physical, mental, elemental, magical), the scope of impact, and the degree of control and mastery exhibited by the individual. This assessment is based on a combination of empirical evidence, expert judgment, and scenario analysis, and is intended to provide a comprehensive picture of the potential consequences of the individual's abilities.

The resulting F-S classification is expressed as a combination of the power level and potential impact, such as "C-R" for an individual with moderate power level and regional potential impact, or "S-G" for an individual with high power level and global potential impact. These classifications are intended to provide a standardized and actionable framework for understanding the diversity of awakened abilities and their implications for society.

It is important to note that the F-S scale is not intended to be a fixed or deterministic system, but rather a dynamic and evolving tool that can be refined and adapted in response to new evidence and insights. As our understanding of awakened abilities continues to grow, it is likely that the F-S scale will need to be updated and expanded to account for new dimensions and complexities.

Moreover, the F-S scale should not be seen as a substitute for more nuanced and context-specific analyses of awakened abilities and their implications. While the scale provides a useful starting point for classifying and comparing different types of abilities, it is important to recognize that the expression and impact of these abilities are shaped by a wide range of individual, social, and environmental factors that cannot be fully captured by a single metric.

Despite these limitations, the F-S scale represents a significant advance in our understanding of awakened abilities and their place in the post-catastrophe world. By providing a standardized and evidence-based framework for classifying these abilities, the scale has the potential to facilitate more rigorous and comparable research, inform policy and decision-making, and ultimately contribute to a more equitable and sustainable future for all.

\section{Research on Awakened Abilities: A Review and Synthesis}
Since the emergence of awakened abilities in the wake of the 2025 catastrophe, a growing body of research has sought to understand the nature, distribution, and implications of these extraordinary powers. In this section, we review and synthesize the existing literature on awakened abilities, highlighting key findings, methodological approaches, and research gaps.

\subsection{Nature and Distribution of Awakened Abilities}
One of the earliest and most influential studies on the nature of awakened abilities was conducted by Yamada and Watanabe \cite{Yamada2025}, who used a combination of surveys, interviews, and laboratory tests to investigate the types and prevalence of abilities among a sample of 5,000 Japanese adults. They found that approximately 3\% of the population had awakened abilities, with a wide range of powers represented, including enhanced physical strength, telepathy, telekinesis, elemental manipulation, and magical abilities.

Subsequent studies have provided additional evidence for the diversity of awakened abilities, as well as their uneven distribution across populations. For example, Kim and Lee \cite{Kim2027} conducted a nationwide survey of awakened individuals in South Korea (n=10,000) and found that the prevalence of abilities varied significantly by region, with higher rates in urban areas and among younger age groups. They also identified several clusters of abilities that tended to co-occur, such as enhanced senses and physical abilities, telepathy and empathy, and elemental manipulation and magical abilities.

Other researchers have focused on the psychological and neurological underpinnings of awakened abilities, seeking to understand the mechanisms by which these powers manifest and develop. Nakamura and Yamamoto \cite{Nakamura2026} used functional magnetic resonance imaging (fMRI) to compare the brain activity of awakened and non-awakened individuals during a series of tasks and found that awakened individuals exhibited heightened activity in regions associated with perception, attention, and cognitive control. They also identified a set of genetic markers that were more common among awakened individuals, suggesting a possible biological basis for these abilities.

Despite these advances, much remains unknown about the nature and distribution of awakened abilities, particularly in terms of their long-term development and the factors that shape their expression and impact. Longitudinal studies that track the experiences and outcomes of awakened individuals over time are needed to better understand the trajectories and consequences of these abilities, as well as the ways in which they interact with other aspects of individuals' lives and identities.

\subsection{Social and Political Implications of Awakened Abilities}
In addition to research on the nature and distribution of awakened abilities, a growing body of literature has examined the social and political implications of these powers for individuals, communities, and societies. One key area of focus has been the role of awakened individuals in shaping the response to the 2025 catastrophe and its aftermath, particularly in terms of their involvement in hunter guilds and other organizations dedicated to exploring and combating the threats posed by dungeons and monsters.

For example, Sato and colleagues \cite{Sato2026} conducted a comparative analysis of the organizational structures and practices of hunter guilds in Japan, South Korea, and the United States, highlighting the ways in which these groups have adapted to the unique challenges and opportunities posed by awakened abilities. They found that guilds with a higher proportion of awakened members tended to have more specialized and hierarchical structures, as well as more formalized systems for training, support, and resource allocation.

Other researchers have examined the broader social and political implications of awakened abilities, particularly in terms of their potential to exacerbate existing inequalities and power imbalances. Chen and Liu \cite{Chen2026} argued that the emergence of awakened abilities has created new forms of social stratification and conflict, as individuals with powerful abilities are often recruited into elite military, corporate, and governmental roles, while those with less potent or socially valued abilities may face discrimination and marginalization.

Similarly, Nakamura and Yamamoto \cite{Nakamura2026b} explored the ethical and legal challenges posed by awakened abilities, particularly in terms of issues of privacy, consent, and responsibility. They argued that existing legal and regulatory frameworks are ill-equipped to deal with the complex and rapidly evolving landscape of awakened abilities, and called for the development of new approaches that prioritize the rights and well-being of all individuals, regardless of their abilities.

\subsection{Gaps and Future Directions}
Despite the growing body of research on awakened abilities, significant gaps and challenges remain. One key issue is the lack of standardization and comparability across studies, which makes it difficult to synthesize findings and draw broader conclusions about the nature and implications of these abilities. The development of common measures, protocols, and reporting standards for research on awakened abilities is thus an important priority for the field.

Another challenge is the need for more diverse and representative samples of awakened individuals, particularly in terms of gender, race, ethnicity, and socioeconomic status. Much of the existing research on awakened abilities has focused on relatively homogeneous and privileged populations, which may limit the generalizability and applicability of the findings.

There is also a need for more interdisciplinary and collaborative research on awakened abilities, drawing on insights and methods from a wide range of fields, including psychology, sociology, anthropology, political science, and biology. Such research can help to provide a more comprehensive and holistic understanding of the complex interplay between individual abilities, social and cultural contexts, and broader structural and institutional factors.

Finally, as noted earlier, there is a critical need for more participatory and action-oriented research on awakened abilities, which centers the perspectives and experiences of awakened individuals themselves. By engaging these individuals as active partners and co-creators of knowledge, researchers can help to ensure that their work is relevant, ethical, and empowering for the communities they seek to understand and serve.

\section{A Typology of Awakened Abilities}
Building on the existing research and the F-S scale framework, we propose a typology of awakened abilities that captures the key dimensions and variations of these powers. Our typology is based on a comprehensive review of the literature, as well as original data from our nationwide survey of awakened individuals in Japan (n=3,000) and in-depth interviews with a subset of respondents (n=100).

\subsection{Physical Abilities}
Physical abilities are those that enhance an individual's bodily strength, speed, endurance, or resilience. These abilities are often associated with heightened sensory acuity, reflexes, and regenerative powers. Examples of physical abilities include:

\begin{itemize}
    \item Enhanced strength: The ability to lift, push, or pull objects of great weight or density.
    \item Enhanced speed: The ability to move or react at superhuman speeds, often accompanied by heightened reflexes and agility.
    \item Enhanced endurance: The ability to sustain physical activity for extended periods without fatigue or injury.
    \item Regeneration: The ability to heal from wounds or injuries at an accelerated rate, sometimes even regrowing lost limbs or organs.
    \item Sensory enhancement: The ability to see, hear, smell, taste, or feel with greater acuity or range than normal human senses.
    \item Invisibility: The ability to render oneself partially or completely invisible to the naked eye.
    \item Shapeshifting: The ability to alter one's physical form or appearance, sometimes taking on the characteristics of other creatures or objects.
\end{itemize}

In our survey, 35\% of respondents reported having at least one physical ability, with enhanced strength, speed, and endurance being the most common (Table \ref{tab:physicalabilities}). Interestingly, we found that physical abilities were more prevalent among male respondents and those in younger age groups, suggesting possible gender and age differences in the expression of these powers.

\begin{table}[h]
\centering
\caption{Prevalence of physical abilities among awakened individuals in Japan (n=3,000)}
\label{tab:physicalabilities}
\begin{tabular}{lr}
\toprule
Ability & Prevalence (\%) \\
\midrule
Enhanced strength & 18.2 \\
Enhanced speed & 15.7 \\
Enhanced endurance & 14.3 \\
Regeneration & 9.8 \\
Sensory enhancement & 8.5 \\
Invisibility & 3.2 \\
Shapeshifting & 1.7 \\
\bottomrule
\end{tabular}
\end{table}

\subsection{Mental Abilities}
Mental abilities are those that enhance an individual's cognitive, emotional, or perceptual capacities. These abilities often involve the manipulation of thoughts, feelings, or sensory information, either within oneself or in others. Examples of mental abilities include:

\begin{itemize}
    \item Telepathy: The ability to read or communicate thoughts or feelings directly from one mind to another.
    \item Empathy: The ability to sense or manipulate the emotions or emotional states of others.
    \item Precognition: The ability to perceive or predict future events through extrasensory means.
    \item Clairvoyance: The ability to perceive or gather information about distant or hidden objects, people, or events.
    \item Psychokinesis: The ability to manipulate physical objects or forces with the power of the mind.
    \item Mind control: The ability to influence or control the thoughts, emotions, or behaviors of others through mental means.
    \item Illusion: The ability to create or manipulate sensory illusions or hallucinations in oneself or others.
    \item Astral projection: The ability to separate one's consciousness from the physical body and travel in a non-corporeal form.
\end{itemize}

In our survey, 32\% of respondents reported having at least one mental ability, with telepathy, empathy, and precognition being the most common (Table \ref{tab:mentalabilities}). We found that mental abilities were more prevalent among female respondents and those in older age groups, suggesting possible gender and age differences in the expression of these powers.

\begin{table}[h]
\centering
\caption{Prevalence of mental abilities among awakened individuals in Japan (n=3,000)}
\label{tab:mentalabilities}
\begin{tabular}{lr}
\toprule
Ability & Prevalence (\%) \\
\midrule
Telepathy & 13.5 \\
Empathy & 11.2 \\
Precognition & 9.7 \\
Clairvoyance & 7.4 \\
Psychokinesis & 6.8 \\
Mind control & 4.3 \\
Illusion & 3.6 \\
Astral projection & 2.1 \\
\bottomrule
\end{tabular}
\end{table}

\subsection{Elemental Abilities}
Elemental abilities are those that involve the manipulation or generation of specific natural elements, such as fire, water, earth, air, or electricity. These abilities often require a deep understanding of the properties and behaviors of the elements in question, as well as the ability to channel and control their power. Examples of elemental abilities include:

\begin{itemize}
    \item Pyrokinesis: The ability to create, control, or manipulate fire.
    \item Hydrokinesis: The ability to create, control, or manipulate water.
    \item Geokinesis: The ability to manipulate earth, stone, or other geological materials.
    \item Aerokinesis: The ability to manipulate air, wind, or other atmospheric phenomena.
    \item Electrokinesis: The ability to generate, control, or manipulate electrical currents or fields.
    \item Cryokinesis: The ability to create, control, or manipulate ice and cold temperatures.
    \item Photokinesis: The ability to generate, control, or manipulate light and photons.
    \item Umbrakinesis: The ability to manipulate shadows and darkness.
\end{itemize}

In our survey, 28\% of respondents reported having at least one elemental ability, with pyrokinesis, hydrokinesis, and geokinesis being the most common (Table \ref{tab:elementalabilities}). We found no significant gender or age differences in the prevalence of elemental abilities, suggesting that these powers may be more evenly distributed across the population.

\begin{table}[h]
\centering
\caption{Prevalence of elemental abilities among awakened individuals in Japan (n=3,000)}
\label{tab:elementalabilities}
\begin{tabular}{lr}
\toprule
Ability & Prevalence (\%) \\
\midrule
Pyrokinesis & 8.4 \\
Hydrokinesis & 7.9 \\
Geokinesis & 6.5 \\
Aerokinesis & 5.7 \\
Electrokinesis & 5.2 \\
Cryokinesis & 3.8 \\
Photokinesis & 3.3 \\
Umbrakinesis & 2.6 \\
\bottomrule
\end{tabular}
\end{table}

\subsection{Magical Abilities}
Magical abilities are those that involve the manipulation of supernatural forces or energies that defy conventional scientific explanation. These abilities often require a deep understanding of the nature of magic and the ability to harness and channel its power through various means, such as incantations, gestures, or the use of magical objects. Examples of magical abilities include:

\begin{itemize}
    \item Enchantment: The ability to imbue objects, people, or spaces with magical properties or effects.
    \item Summoning: The ability to call forth spirits, creatures, or objects from other planes of existence.
    \item Necromancy: The ability to manipulate the forces of life and death, often involving the control of undead creatures or the manipulation of life energy.
    \item Divination: The ability to gather information or insights through supernatural means, such as scrying, prophecy, or communication with otherworldly entities.
    \item Alchemy: The ability to transmute or manipulate matter through magical means, often involving the creation of potions, elixirs, or other supernatural substances.
    \item Blessing/Curse: The ability to bestow positive or negative effects on others through magical means, often involving the invocation of higher powers or the manipulation of fate.
    \item Illusion magic: The ability to create or manipulate complex and realistic illusions that can deceive the senses and the mind.
    \item Teleportation: The ability to instantly transport oneself or others across distances through magical means.
\end{itemize}

In our survey, 20\% of respondents reported having at least one magical ability, with enchantment, summoning, and necromancy being the most common (Table \ref{tab:magicalabilities}). We found that magical abilities were more prevalent among younger respondents and those with higher levels of education, suggesting possible developmental and learning factors in the expression of these powers.

\begin{table}[h]
\centering
\caption{Prevalence of magical abilities among awakened individuals in Japan (n=3,000)}
\label{tab:magicalabilities}
\begin{tabular}{lr}
\toprule
Ability & Prevalence (\%) \\
\midrule
Enchantment & 6.7 \\
Summoning & 5.2 \\
Necromancy & 4.8 \\
Divination & 4.1 \\
Alchemy & 3.6 \\
Blessing/Curse & 3.3 \\
Illusion magic & 2.9 \\
Teleportation & 2.4 \\
\bottomrule
\end{tabular}
\end{table}

\subsection{Hybrid Abilities}
In addition to the four main categories of physical, mental, elemental, and magical abilities, our research also identified a significant number of hybrid abilities that combine elements of two or more categories. These abilities often involve complex and synergistic interactions between different powers, resulting in unique and unpredictable effects. Examples of hybrid abilities include:

\begin{itemize}
    \item Psionic manipulation: The ability to manipulate physical objects or forces using a combination of mental and elemental powers, such as telekinetically controlling fire or water.
    \item Elemental enhancement: The ability to use elemental powers to enhance one's physical abilities, such as using air manipulation to increase speed or agility.
    \item Sensory projection: The ability to use mental powers to extend or project one's sensory abilities, such as using clairvoyance to see through walls or telepathy to communicate over long distances.
    \item Mana manipulation: The ability to sense, absorb, or manipulate the magical energy or life force that pervades the universe, often used to fuel or enhance other abilities.
    \item Technopathy: The ability to mentally control or communicate with technological devices or systems, often involving a combination of mental and elemental powers.
    \item Shapeshifting magic: The ability to use magical means to alter one's physical form or appearance, often involving the manipulation of elemental forces or the invocation of supernatural entities.
    \item Astral combat: The ability to engage in combat or conflict on the astral plane, often involving a combination of mental, magical, and physical powers.
\end{itemize}

In our survey, 15\% of respondents reported having at least one hybrid ability, with psionic manipulation and elemental enhancement being the most common (Table \ref{tab:hybridabilities}). We found that hybrid abilities were more prevalent among younger respondents and those with higher levels of education, suggesting possible developmental and learning factors in the expression of these powers.

\begin{table}[h]
\centering
\caption{Prevalence of hybrid abilities among awakened individuals in Japan (n=3,000)}
\label{tab:hybridabilities}
\begin{tabular}{lr}
\toprule
Ability & Prevalence (\%) \\
\midrule
Psionic manipulation & 5.6 \\
Elemental enhancement & 4.8 \\
Sensory projection & 3.9 \\
Mana manipulation & 3.5 \\
Technopathy & 3.1 \\
Shapeshifting magic & 2.7 \\
Astral combat & 2.3 \\
\bottomrule
\end{tabular}
\end{table}

\subsection{Mana-Related Abilities}
In addition to the abilities mentioned above, our research also identified a subset of abilities that specifically involve the manipulation or utilization of mana, a form of supernatural energy or life force that is believed to permeate the universe. Mana-related abilities are often seen as a distinct category of powers, as they involve the direct manipulation of this fundamental energy rather than its manifestation through specific elements or phenomena. Examples of mana-related abilities include:

\begin{itemize}
    \item Mana sensing: The ability to detect or perceive the presence and flow of mana in oneself, others, or the environment.
    \item Mana absorption: The ability to absorb or draw in mana from external sources, such as other living beings, magical objects, or the environment itself.
    \item Mana channeling: The ability to direct or focus the flow of mana within oneself or others, often used to empower or enhance other abilities.
    \item Mana projection: The ability to project or expel mana from one's body, often in the form of energy blasts, barriers, or other constructs.
    \item Mana healing: The ability to use mana to heal or regenerate oneself or others, often by channeling the energy into the affected areas.
    \item Mana disruption: The ability to disrupt or interfere with the flow of mana in others or in the environment, often used to weaken or negate other abilities.
    \item Mana storage: The ability to store or accumulate excess mana within one's body or in external objects for later use.
\end{itemize}

In our survey, 12\% of respondents reported having at least one mana-related ability, with mana sensing and mana channeling being the most common (Table \ref{tab:manaabilities}). We found that mana-related abilities were more prevalent among female respondents and those with higher levels of spiritual or religious practice, suggesting possible gender and cultural factors in the expression of these powers.

\begin{table}[h]
\centering
\caption{Prevalence of mana-related abilities among awakened individuals in Japan (n=3,000)}
\label{tab:manaabilities}
\begin{tabular}{lr}
\toprule
Ability & Prevalence (\%) \\
\midrule
Mana sensing & 5.8 \\
Mana channeling & 4.6 \\
Mana absorption & 3.9 \\
Mana projection & 3.5 \\
Mana healing & 3.1 \\
Mana disruption & 2.7 \\
Mana storage & 2.4 \\
\bottomrule
\end{tabular}
\end{table}

\section{Limitations and Directions for Future Research}
While our typology of awakened abilities represents a significant advance in the field, it is important to acknowledge its limitations and the need for further research. One key limitation of our study is that it relies primarily on self-reported data from a single country, Japan. While Japan has been at the forefront of research on awakened abilities since the 2025 catastrophe, it is possible that the prevalence and distribution of abilities may differ in other cultural and geographic contexts. Future research should aim to replicate and extend our findings in other countries and regions, using a variety of methodological approaches.

Another limitation of our study is that it focuses primarily on the individual level of analysis, examining the prevalence and characteristics of specific abilities among awakened individuals. While this approach provides valuable insights into the nature and diversity of awakened abilities, it does not fully capture the complex social, cultural, and political contexts in which these abilities are expressed and regulated. Future research should aim to integrate individual-level data with more macro-level analyses of the social and institutional factors that shape the development and impact of awakened abilities, such as the role of hunter guilds, government policies, and public attitudes.

In addition, our study is limited by its cross-sectional design, which provides a snapshot of the prevalence and distribution of abilities at a single point in time. To fully understand the developmental trajectories and long-term consequences of awakened abilities, longitudinal studies that track individuals over time are needed. Such studies could provide valuable insights into the factors that influence the emergence and evolution of abilities, as well as the ways in which abilities shape individuals' life courses and outcomes.

Finally, as noted earlier, our study highlights the need for more participatory and action-oriented research on awakened abilities, which centers the perspectives and experiences of awakened individuals themselves. While our survey and interviews provide valuable insights into the lived experiences of awakened individuals, they are ultimately filtered through the lens of the researchers. Future research should aim to create more opportunities for awakened individuals to shape the research agenda, design, and dissemination process, in order to ensure that the knowledge produced is relevant, empowering, and actionable for their communities.

\section{Conclusion}
The emergence of awakened abilities in the wake of the 2025 catastrophe has profound implications for individuals, communities, and societies around the world. As these abilities continue to evolve and spread, it is essential that we develop a more comprehensive and nuanced understanding of their nature, prevalence, and impact. Our typology of awakened abilities, based on the F-S scale and a comprehensive review of the literature, represents a significant step forward in this regard.

By identifying the key dimensions and categories of abilities, including physical, mental, elemental, magical, and hybrid abilities, as well as mana-related abilities, our typology provides a framework for classifying and comparing the wide range of powers that have emerged in recent years. Our findings also highlight the importance of considering the complex interplay of individual, social, and cultural factors in shaping the expression and impact of these abilities.

However, our study also underscores the need for further research and action to address the challenges and opportunities posed by awakened abilities. This includes the need for more interdisciplinary and participatory research that centers the perspectives and experiences of awakened individuals, as well as the development of policies and practices that promote the responsible and equitable use of these powers for the benefit of all.

Ultimately, the emergence of awakened abilities represents a transformative moment in human history, one that calls for a fundamental rethinking of our understanding of the nature of human potential and the possibilities for individual and collective flourishing. By embracing this moment with curiosity, compassion, and creativity, we have the opportunity to shape a future in which the extraordinary abilities of the few are harnessed for the benefit of the many, and in which the full potential of every individual is recognized and celebrated.

\bibliographystyle{apalike}
\bibliography{references}

\end{document}
