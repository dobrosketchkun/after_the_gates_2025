\documentclass[12pt, a4paper]{article}
\usepackage[margin=1in]{geometry}
\usepackage{graphicx}
\usepackage{natbib}
\usepackage[utf8]{inputenc}
\usepackage{amsmath}
\usepackage{amssymb}
\usepackage{listings}

\title{The Emergence of Supernatural Terrorism and Its Impact on Global Security}
\author{Xiaowei Chen\textsuperscript{1}and Jingyi Liu\textsuperscript{2}}
\date{April 2026}


\begin{document}

\maketitle

 \textsuperscript{1}Department of Supernatural Security Studies, Tsinghua University, Beijing, China\\
 \textsuperscript{2}School of International Relations and Public Affairs, Fudan University, Shanghai, China

\begin{abstract}
This study investigates the emergence of supernatural terrorism in the post-catastrophe world. The authors examine the motivations, tactics, and impact of terrorist groups that seek to exploit the powers of awakened individuals for ideological, political, or religious purposes. The findings highlight the unique challenges posed by supernatural terrorism, such as the difficulty of detecting and countering abilities that defy conventional security measures, the psychological impact of attacks that blur the line between the natural and the supernatural, and the potential for escalation and arms races among state and non-state actors. The paper discusses the need for international cooperation, innovative security strategies, and counter-radicalization efforts to address the threat of supernatural terrorism.
\end{abstract}

\section{Introduction}
The catastrophic events of 2025 marked a turning point in human history, as the emergence of portals leading to treacherous environments, known as dungeons, reshaped the global landscape. These dungeons, filled with powerful creatures and resources, gave rise to a new class of individuals with extraordinary abilities, known as the awakened. While the majority of the awakened have joined hunter guilds or collaborated with governments to explore and manage the challenges posed by the dungeons, a small but significant number have turned to terrorism, exploiting their powers for ideological, political, or religious purposes.

Supernatural terrorism, defined as the use of awakened abilities to instill fear and achieve political or ideological goals, has emerged as a major threat to global security in the post-catastrophe world. The unique nature of awakened abilities, which often defy conventional security measures and blur the line between the natural and the supernatural, has made it difficult for governments and international organizations to detect and counter these threats \citep{Hoffman2026}.

The rise of supernatural terrorism has been fueled by a complex interplay of factors, including the psychological impact of the catastrophe, the uneven distribution of awakened abilities, and the ideological and religious interpretations of the dungeons and their implications for human society \citep{Nakamura2027}. Some terrorist groups have sought to frame the emergence of the dungeons as a divine intervention or a harbinger of the end times, using these narratives to justify violence and recruit new members among the awakened population.

The impact of supernatural terrorism has been far-reaching, with attacks on civilian populations, critical infrastructure, and government institutions causing widespread fear, economic disruption, and political instability \citep{Muller2027}. The use of awakened abilities in these attacks, such as telekinesis, elemental manipulation, and mind control, has challenged traditional notions of security and warfare, leading to a new era of asymmetric conflicts and unconventional threats.

In this paper, we examine the emergence of supernatural terrorism and its impact on global security in the post-catastrophe world. We begin by providing an overview of the key concepts and definitions related to supernatural terrorism and the awakened population. We then analyze the motivations and tactics of supernatural terrorist groups, drawing upon case studies and intelligence reports from around the world. Next, we assess the impact of supernatural terrorism on global security, focusing on the challenges posed by these threats and the potential for escalation and arms races among state and non-state actors. Finally, we discuss the need for international cooperation, innovative security strategies, and counter-radicalization efforts to address the threat of supernatural terrorism, and we offer policy recommendations for governments and international organizations.

\section{Background and Key Concepts}
\subsection{The 2025 Catastrophe and the Emergence of Dungeons}
The global catastrophe of 2025 marked a watershed moment in human history, as the sudden appearance of portals leading to mysterious, enclosed environments, known as dungeons, reshaped the world as we knew it. These dungeons, which vary in size, complexity, and threat level, are filled with powerful creatures, anomalous phenomena, and valuable resources that defy conventional scientific understanding \citep{Sørensen2026}.

The dungeons are classified according to a standard system, ranging from F-rank, the least threatening, to S-rank, the most dangerous and complex. Each dungeon is a self-contained environment with its own unique ecosystem, physical laws, and challenges. Some dungeons resemble natural environments, such as caves, forests, or underwater realms, while others are more akin to urban or industrial settings, such as abandoned cities or factory complexes \citep{Nakano2027}.

One of the most striking features of the dungeons is their one-way nature. While it is possible to enter a dungeon through its corresponding portal, exiting the dungeon is only possible by either clearing its objectives, which typically involves defeating a powerful boss creature, or by using rare and expensive magical items. This one-way nature has made dungeon exploration a risky and challenging endeavor, requiring careful planning, teamwork, and specialized equipment \citep{Iwasaki2026}.

The emergence of the dungeons has had a profound impact on human society, transforming the global economy, politics, and culture in ways that are still unfolding. Governments, corporations, and hunter guilds have sought to explore and exploit the dungeons for their resources and knowledge, leading to a new era of adventure, conflict, and innovation. At the same time, the dungeons have also given rise to new threats and challenges, including the emergence of supernatural terrorism, which seeks to exploit the power of the awakened for destructive ends.

\subsection{The Awakened Population and Hunter Guilds}
One of the most significant consequences of the 2025 catastrophe has been the emergence of a small but significant population of individuals with extraordinary abilities, known as the awakened. These abilities, which range from enhanced physical strength and speed to elemental manipulation and psychic powers, are believed to be the result of exposure to the anomalous energy fields generated by the dungeons \citep{Nakamura2026}.

The awakened population is estimated to comprise less than 1\% of the global population, with the majority of awakened individuals concentrated in regions with a high density of dungeons. The distribution of abilities among the awakened is highly variable, with some individuals possessing only minor enhancements while others wield godlike powers. Like the dungeons themselves, awakened individuals are classified according to a standard system, ranging from F-rank to S-rank, based on the strength and versatility of their abilities \citep{Kim2027}.

The emergence of the awakened has led to the rise of hunter guilds, organizations that specialize in recruiting, training, and deploying awakened individuals for dungeon exploration and monster hunting. These guilds, which range from small, local outfits to large, international franchises, have become major players in the post-catastrophe world, providing essential services and support to governments, corporations, and communities \citep{Liu2026}.

Hunter guilds operate according to a variety of business models, with some charging fees for their services while others rely on sponsorships, donations, or government subsidies. Many guilds also engage in the sale of dungeon resources and artifacts, which have become highly sought-after commodities in the post-catastrophe economy. The most successful guilds are known for their skilled and experienced hunters, as well as their access to advanced technology and magical equipment.

Despite their important role in the post-catastrophe world, hunter guilds have also been criticized for their lack of regulation and accountability, as well as their potential to exacerbate social inequalities and political tensions. Some guilds have been accused of engaging in unethical or illegal practices, such as exploiting awakened individuals or collaborating with criminal organizations. The rise of supernatural terrorism has only added to these concerns, as some terrorist groups have sought to infiltrate or coopt hunter guilds for their own purposes.

\subsection{Defining Supernatural Terrorism}
Supernatural terrorism is a relatively new and evolving concept that has emerged in the wake of the 2025 catastrophe and the rise of the awakened population. While there is no universally accepted definition of supernatural terrorism, most experts agree that it involves the use of awakened abilities to instill fear and achieve political or ideological goals \citep{Hoffman2026}.

Unlike conventional terrorism, which relies on physical violence and intimidation, supernatural terrorism exploits the unique and often terrifying nature of awakened abilities to create a sense of helplessness and despair among targeted populations. Supernatural terrorist attacks can take many forms, from the use of mind control to manipulate or deceive, to the unleashing of destructive elemental forces in populated areas.

One of the key challenges in defining and studying supernatural terrorism is the lack of clear boundaries between the natural and the supernatural in the post-catastrophe world. Many awakened abilities defy conventional scientific understanding and challenge traditional notions of cause and effect. This blurring of the lines between the real and the imagined has made it difficult for authorities to distinguish between genuine supernatural threats and hoaxes or delusions.

Another challenge in defining supernatural terrorism is the diversity of motivations and ideologies that drive these groups. While some supernatural terrorist organizations are motivated by religious or apocalyptic beliefs, others are driven by political or ethnic grievances, or by a desire for power and control. This diversity has made it difficult to develop a coherent and comprehensive strategy for countering supernatural terrorism.

Despite these challenges, there is a growing consensus among security experts and policymakers that supernatural terrorism represents a significant and urgent threat to global security. The unique and unpredictable nature of awakened abilities, combined with the psychological impact of supernatural attacks, has the potential to destabilize societies and undermine public trust in institutions. As such, developing effective strategies for preventing, detecting, and responding to supernatural terrorism has become a top priority for governments and international organizations around the world.

\section{Motivations and Tactics of Supernatural Terrorist Groups}
\subsection{Ideological and Religious Motivations}
One of the primary motivations behind supernatural terrorism is the desire to promote or defend a particular ideology or religious belief system. Many supernatural terrorist groups have emerged from existing extremist or fundamentalist movements, seeking to use the power of the awakened to advance their cause or bring about a desired political or social change \citep{Nakamura2027}.

For example, some religious extremist groups have interpreted the emergence of the dungeons and the awakened as signs of divine intervention or the end times, using these beliefs to justify violence and recruit new members. These groups often frame their actions as a holy war or a cosmic struggle between good and evil, with the awakened serving as soldiers or martyrs in this conflict \citep{Hoffman2026}.

Other ideological motivations for supernatural terrorism include nationalist or separatist movements, anti-government or anarchist groups, and eco-terrorist organizations. These groups may see the awakened as a means to level the playing field against more powerful adversaries, or to strike fear into the hearts of their enemies and the general public.

\subsection{Strategic and Tactical Considerations}
In addition to ideological and religious motivations, supernatural terrorist groups also engage in strategic and tactical planning to maximize the impact and effectiveness of their attacks. These groups often seek to exploit the unique capabilities of the awakened to achieve specific objectives, such as causing mass casualties, disrupting critical infrastructure, or undermining public trust in institutions \citep{Chen2028}.

One common tactic employed by supernatural terrorist groups is the use of awakened sleeper agents or infiltrators to gather intelligence, sabotage operations, or carry out surprise attacks. These individuals may pose as ordinary civilians or even join hunter guilds or government agencies to gain access to sensitive information or resources.

Another tactic is the use of coordinated, multi-pronged attacks that combine conventional and supernatural methods to overwhelm defenses and cause maximum chaos and confusion. For example, a supernatural terrorist group may use mind control abilities to manipulate security personnel while simultaneously detonating explosives or releasing toxic substances in populated areas.

Supernatural terrorist groups may also seek to exploit the psychological impact of their attacks by staging dramatic or symbolic demonstrations of their power. This could include the use of illusions or hallucinations to create a sense of terror or despair, or the targeting of iconic landmarks or cultural symbols to undermine national morale.

\subsection{Case Studies and Examples}
To illustrate the motivations and tactics of supernatural terrorist groups, we present two brief case studies based on publicly available information and intelligence reports.

The first case involves the so-called "Cult of the Crimson Dawn," a religious extremist group that emerged in the aftermath of the 2025 catastrophe. Led by a charismatic awakened individual known only as the "Red Prophet," the cult believes that the dungeons are gateways to a higher plane of existence and that the awakened are the chosen ones destined to ascend to godhood. The cult has carried out a series of attacks on civilian and government targets, using a combination of supernatural abilities and conventional weapons to sow fear and chaos. In one particularly gruesome incident, cult members used mind control to force a group of schoolchildren to detonate themselves in a crowded shopping mall, killing hundreds of innocent bystanders \citep{Muller2027}.

The second case involves a nationalist separatist group known as the "Free Awakened Army" (FAA), which seeks to establish an independent state for the awakened population in a disputed border region. The FAA has engaged in a campaign of guerrilla warfare and terrorism against government forces and rival hunter guilds, using hit-and-run tactics and surprise attacks to compensate for their relatively small numbers. In one notorious incident, FAA operatives infiltrated a government research facility and released a swarm of genetically engineered monster insects, causing widespread panic and destruction \citep{Liu2026}.

These case studies demonstrate the diversity and complexity of the supernatural terrorist threat, as well as the challenges facing governments and security forces in preventing and responding to these attacks. They also highlight the need for a comprehensive and multi-faceted approach to counter-terrorism that takes into account the unique capabilities and motivations of supernatural terrorist groups.

\section{Impact on Global Security}
\subsection{Challenges to Conventional Security Measures}
The emergence of supernatural terrorism has posed significant challenges to conventional security measures and frameworks. Many of the tools and strategies developed to combat traditional forms of terrorism, such as surveillance, border controls, and physical barriers, are of limited effectiveness against awakened individuals who can teleport, shapeshift, or manipulate minds \citep{Nakamura2027}.

Moreover, the unpredictable and often invisible nature of awakened abilities has made it difficult for security forces to detect and prevent supernatural terrorist attacks. Unlike conventional weapons or explosives, which can be identified and intercepted through physical searches or scanners, awakened abilities can be concealed and activated with little or no warning \citep{Hoffman2026}.

This has led to a heightened sense of vulnerability and uncertainty among both security professionals and the general public, as the threat of supernatural terrorism can seem omnipresent and impossible to guard against. This psychological impact can be just as damaging as the physical effects of an attack, eroding public trust in institutions and leading to a sense of helplessness and despair.

\subsection{Potential for Escalation and Arms Races}
Another major concern surrounding supernatural terrorism is the potential for escalation and arms races among state and non-state actors. As governments and hunter guilds scramble to develop new technologies and strategies to counter the awakened threat, there is a risk that this could lead to a dangerous cycle of innovation and one-upmanship \citep{Chen2028}.

For example, some countries have begun to experiment with the use of awakened individuals in their own military and intelligence operations, raising concerns about the militarization of the awakened population and the blurring of lines between legitimate security forces and supernatural terrorist groups. There is also a risk that the proliferation of advanced anti-awakened weapons and technologies could lead to their acquisition by terrorist groups or other malicious actors.

Moreover, the high stakes and existential nature of the supernatural terrorist threat could lead to a breakdown of international norms and agreements, as countries pursue their own narrow interests at the expense of global cooperation and stability. This could include the use of pre-emptive strikes, covert operations, or even the development of awakened weapons of mass destruction.

\subsection{Implications for International Cooperation and Governance}
Given the transnational and existential nature of the supernatural terrorist threat, effective responses will require unprecedented levels of international cooperation and coordination. This will involve not only traditional security and intelligence sharing agreements, but also new frameworks for regulating the use and development of awakened technologies and abilities \citep{Muller2027}.

One potential model for this is the creation of a global governing body or institution specifically dedicated to addressing the challenges posed by the dungeons and the awakened population. This could include the establishment of international standards and guidelines for the classification and management of awakened individuals, as well as mechanisms for monitoring and enforcing compliance with these standards.

Another key area for international cooperation is the development of counter-radicalization and deradicalization programs aimed at preventing the recruitment and indoctrination of awakened individuals into terrorist groups. This could involve a combination of education, counseling, and social support services, as well as efforts to promote alternative narratives and value systems that reject violence and extremism.

Ultimately, the success of these efforts will depend on the willingness of countries and other stakeholders to put aside their differences and work together in the face of a common threat. This will require a fundamental shift in the way we think about national sovereignty, security, and the role of international institutions in the post-catastrophe world.

\section{Conclusion and Recommendations}
The emergence of supernatural terrorism in the aftermath of the 2025 catastrophe represents a significant and urgent threat to global security, one that challenges conventional notions of warfare, diplomacy, and governance. As we have seen, the unique and often terrifying capabilities of awakened terrorists, combined with their ideological and strategic motivations, have the potential to cause widespread destruction, fear, and instability.

To address this threat, we must develop a comprehensive and multi-faceted approach that encompasses both short-term tactical responses and long-term strategic initiatives. This will require unprecedented levels of international cooperation, innovation, and adaptation, as well as a willingness to confront difficult questions about the nature of power, security, and human identity in the post-catastrophe world.

Based on the findings of this study, we offer the following recommendations for policymakers, security professionals, and other stakeholders:

1. Develop a global framework for the classification, regulation, and management of awakened individuals and technologies, including standards for the use of awakened abilities in military and security operations.

2. Invest in research and development of new technologies and strategies for detecting, preventing, and responding to supernatural terrorist attacks, including advanced surveillance and early warning systems, as well as countermeasures against specific awakened abilities.

3. Expand international intelligence and security cooperation, including the sharing of information and best practices related to supernatural terrorism, as well as joint training and operations among hunter guilds and security forces.

4. Implement comprehensive counter-radicalization and deradicalization programs aimed at preventing the recruitment and indoctrination of awakened individuals into terrorist groups, including education, counseling, and social support services.

5. Foster public trust and resilience through transparent and effective risk communication, as well as efforts to promote social cohesion and community-based responses to the challenges posed by the dungeons and the awakened population.

Ultimately, the fight against supernatural terrorism will require a sustained and collective effort that transcends national and ideological boundaries. Only by working together can we hope to build a more secure, just, and resilient world in the face of this unprecedented challenge. It is our hope that this study will contribute to this effort by providing a deeper understanding of the motivations, tactics, and impact of supernatural terrorist groups, as well as the strategies and approaches needed to counter them.

\bibliographystyle{apalike}
\bibliography{references}

\end{document}
