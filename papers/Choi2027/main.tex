\documentclass[12pt, a4paper]{article}

\usepackage[margin=1in]{geometry}
\usepackage{graphicx}
\usepackage{natbib}
\usepackage[utf8]{inputenc}
\usepackage{amsmath}
\usepackage{amssymb}
\usepackage{siunitx}

\title{Mana and Awakened Abilities: \\Developing a Comprehensive Assessment Framework}

\author{Jun-Young Choi\textsuperscript{1} \and Hye-Soo Park\textsuperscript{2} \\
\small \textsuperscript{1}Department of Mana and Ability Studies, Seoul National University, Seoul, South Korea\\
\small \textsuperscript{2}Research Institute for Awakened Phenomena, Korea Advanced Institute \\ \small \textsuperscript{2}of Science and Technology, Daejeon, South Korea}

\date{August 2027}

\begin{document}

\maketitle

\begin{abstract}
In the wake of the 2025 catastrophe, which saw the emergence of dungeon portals and the awakening of a small portion of the population with extraordinary abilities, the need for a comprehensive assessment framework for mana and awakened abilities has become increasingly pressing. This paper proposes a standardized set of measures based on the Fujita-Sato (F-S) scale, which categorizes both dungeons and awakened individuals according to their power levels. We review existing research on the nature and distribution of mana and awakened abilities, identify key research gaps, and outline a psychometric approach for developing and validating the proposed measures. Our findings highlight the importance of a reliable and valid assessment framework for advancing our understanding of these phenomena and informing the training and management of awakened individuals.
\end{abstract}

\section{Introduction}

The 2025 catastrophe marked a turning point in human history, as the sudden appearance of dungeon portals and the awakening of a small portion of the population with extraordinary abilities reshaped the global landscape. These dungeons, ranging from underground caves to abandoned cities and lava-infested hellscapes, are filled with powerful monsters and valuable resources, accessible only through one-way portals that have emerged across the world \citep{Nakamura2026}. Concurrently, a minority of individuals have awakened to possess unique abilities, often involving the manipulation of mana, a mysterious energy source \citep{Kimura2026}.

The emergence of these phenomena has given rise to a new class of individuals known as hunters, who possess the skills and abilities necessary to navigate the treacherous dungeons and combat the monsters within \citep{Sato2027}. Hunters are typically awakened individuals who have honed their abilities through training and experience, and are often organized into guilds that provide support and resources for their members \citep{Watanabe2027}.

To facilitate the assessment and management of these unprecedented developments, researchers have proposed various classification systems for dungeons and awakened abilities. One of the most widely used is the Fujita-Sato (F-S) scale, which categorizes dungeons and awakened individuals into ranks ranging from F to S, based on their power levels and potential impact \citep{Fujita2026}. However, the F-S scale is primarily a descriptive tool and lacks the standardization and validation necessary for rigorous assessment and comparison across different contexts.

The purpose of this paper is to propose a comprehensive assessment framework for mana and awakened abilities, building upon the F-S scale and incorporating psychometric principles to develop standardized measures. We begin by reviewing the existing literature on the nature and distribution of mana and awakened abilities, with a focus on the F-S scale and its applications. We then outline our proposed assessment framework, which includes a set of standardized measures for evaluating the power levels and characteristics of awakened individuals, as well as guidelines for administering and interpreting these measures. Finally, we discuss the implications of our proposed framework for research and practice, and identify key challenges and future directions for the field.

\section{Literature Review}

\subsection{The Nature of Mana and Awakened Abilities}

Mana is a mysterious energy source that has become increasingly prevalent since the emergence of dungeon portals in 2025. While the exact nature of mana remains poorly understood, researchers have identified several key characteristics that distinguish it from other forms of energy \citep{Nakamura2026, Kimura2026}. First, mana appears to be closely linked to the dungeons themselves, with higher concentrations of mana found in more powerful and dangerous dungeons. Second, mana can be manipulated by certain awakened individuals, who are able to harness its power to perform extraordinary feats, such as enhanced strength, speed, and sensory abilities \citep{Sato2027}.

The distribution of awakened abilities among the population remains a topic of ongoing research. Initial studies suggested that around 1\% of the global population had awakened to possess some form of extraordinary ability following the 2025 catastrophe \citep{Fujita2026}. However, more recent estimates suggest that the true prevalence may be closer to 0.1\%, with significant variations across different regions and demographic groups \citep{Choi2027}.

Despite the rarity of awakened individuals, their abilities have had a profound impact on society, particularly in the realm of dungeon exploration and monster combat. Awakened individuals who become hunters are able to leverage their abilities to navigate the dangers of the dungeons and extract valuable resources, which has led to the emergence of a new class of elite warriors and explorers \citep{Watanabe2027}.

\subsection{The Fujita-Sato (F-S) Scale}

The Fujita-Sato (F-S) scale is a widely used classification system for dungeons and awakened individuals, developed by researchers at the University of Tokyo in the aftermath of the 2025 catastrophe \citep{Fujita2026}. The scale assigns ranks to dungeons and individuals based on their power levels, with F being the lowest rank and S being the highest.

For dungeons, the F-S scale takes into account factors such as the strength and rarity of the monsters within, the value of the resources available, and the overall difficulty of navigation and survival. F-rank dungeons are considered relatively low-risk, with weak monsters and few valuable resources, while S-rank dungeons are extremely dangerous, with powerful monsters and abundant rare materials \citep{Nakamura2026}.

For awakened individuals, the F-S scale evaluates the strength and versatility of their abilities, as well as their overall combat prowess and potential impact. F-rank awakened individuals possess relatively weak or specialized abilities, while S-rank individuals are extremely rare and possess abilities of world-changing significance \citep{Sato2027}.

One of the key strengths of the F-S scale is its simplicity and intuitive appeal, which has contributed to its widespread adoption by researchers, hunters, and policymakers. However, the scale has also been criticized for its lack of standardization and validation, as well as its reliance on subjective judgments and informal assessments \citep{Kim2026}.

\subsection{Psychometric Approaches to Assessment}

Psychometric approaches to assessment involve the use of standardized measures and statistical techniques to evaluate the characteristics and abilities of individuals. These approaches have a long history in fields such as psychology, education, and human resources, and have been used to develop a wide range of assessments, from intelligence tests to personality inventories \citep{Furr2014}.

In the context of mana and awakened abilities, psychometric approaches offer a promising avenue for developing more rigorous and reliable assessments. By creating standardized measures of mana manipulation, combat prowess, and other relevant abilities, researchers can more accurately evaluate the power levels and characteristics of awakened individuals, and compare them across different contexts \citep{Choi2027}.

However, the development of psychometrically sound measures of mana and awakened abilities presents several challenges. First, the nature of these abilities is still poorly understood, and there is a lack of consensus on the key dimensions and indicators that should be assessed. Second, the low prevalence of awakened individuals makes it difficult to obtain large and representative samples for validation studies. Finally, the high-stakes nature of dungeon exploration and monster combat raises ethical concerns about the use of assessments for selection and evaluation purposes \citep{Park2027}.

Despite these challenges, there is a growing recognition among researchers and practitioners of the need for more standardized and validated measures of mana and awakened abilities. In the following section, we outline our proposed assessment framework, which seeks to address these challenges and provide a comprehensive approach to evaluating these phenomena.

\section{Proposed Assessment Framework}

\subsection{Overview}

Our proposed assessment framework for mana and awakened abilities builds upon the existing F-S scale, while incorporating psychometric principles to develop standardized measures and procedures. The framework consists of three main components:\\

1. A set of standardized measures for assessing mana manipulation, combat prowess, and other relevant abilities. \\
2. Guidelines for administering and interpreting these measures, including procedures for ensuring reliability and validity. \\
3. A revised version of the F-S scale, based on the results of the standardized assessments, which provides a more accurate and reliable classification of awakened individuals and dungeons.

Table \ref{fig:framework} provides an overview of the proposed assessment framework, illustrating the key components and their interrelationships.

\begin{table}[ht]
\centering
\begin{tabular}{|p{0.3\textwidth}|p{0.6\textwidth}|}
\hline
\textbf{Component} & \textbf{Description} \\
\hline
Standardized Measures & A set of reliable and valid tests for assessing mana manipulation, combat prowess, and other relevant abilities of awakened individuals. \\
\hline
Administration and Interpretation Guidelines & Protocols for ensuring consistent and accurate administration of the measures, as well as evidence-based interpretation of the results. \\
\hline
Revised F-S Scale & An updated version of the Fujita-Sato scale, incorporating quantitative criteria and multidimensional profiles based on the standardized assessment results. \\
\hline
Validation Studies & Empirical studies conducted to establish the reliability, construct validity, and criterion validity of the proposed assessment framework. \\
\hline
Implementation and Use & The application of the assessment framework in real-world contexts, such as hunter guild recruitment, training, and mission planning, guided by principles of fairness, transparency, and accountability. \\
\hline
\end{tabular}
\caption{Overview of the proposed assessment framework for mana and awakened abilities, highlighting the key components and their interrelationships.}
\label{fig:framework}
\end{table}

\subsection{Standardized Measures}

The core of the proposed assessment framework is a set of standardized measures for evaluating mana manipulation, combat prowess, and other relevant abilities. These measures are designed to be reliable, valid, and sensitive to the full range of abilities found among awakened individuals.

\subsubsection{Mana Manipulation}

Mana manipulation refers to the ability of awakened individuals to harness and control mana energy for various purposes, such as enhancing physical abilities, casting spells, or creating magical effects. To assess mana manipulation, we propose a battery of tests that evaluate the following dimensions:

- Mana sensitivity: The ability to detect and perceive mana energy in the environment.\\
- Mana control: The ability to manipulate and direct mana energy with precision and accuracy.\\
- Mana capacity: The maximum amount of mana energy that an individual can store and utilize. \\
- Mana efficiency: The ability to use mana energy efficiently and effectively, minimizing waste and maximizing impact.\\

Table \ref{tab:mana} provides an overview of the proposed tests for assessing each of these dimensions, along with their key features and scoring procedures.

\begin{table}[ht]
\centering
\begin{tabular}{p{0.2\textwidth}|p{0.4\textwidth}|p{0.3\textwidth}} 
 \hline
 Dimension & Test Description & Scoring \\ 
 \hline
 Mana sensitivity & Mana detection task: Participants are asked to identify the presence and intensity of mana energy in a series of stimuli, ranging from low to high concentrations. & Accuracy and reaction time are measured and combined into a sensitivity score. \\ 
 \hline
 Mana control & Mana manipulation task: Participants are asked to manipulate a stream of mana energy to hit a series of targets with varying sizes and distances. & Precision and speed are measured and combined into a control score. \\
 \hline
 Mana capacity & Mana storage task: Participants are asked to absorb and store as much mana energy as possible from a standardized source, until they reach their maximum capacity. & The total amount of mana energy stored is measured and recorded as the capacity score. \\
 \hline
 Mana efficiency & Mana utilization task: Participants are asked to use their stored mana energy to power a standardized magical device for as long as possible. & The duration of the device's operation is measured and recorded as the efficiency score. \\
 \hline
\end{tabular}
\caption{Proposed tests for assessing mana manipulation abilities, along with their key features and scoring procedures.}
\label{tab:mana}
\end{table}

\subsubsection{Combat Prowess}

Combat prowess refers to the overall fighting ability of awakened individuals, including their physical strength, speed, endurance, and combat skills. To assess combat prowess, we propose a series of standardized combat scenarios that evaluate the following dimensions:

- Offensive power: The ability to deal damage to opponents and monsters using physical attacks, weapons, or magical abilities.\\
- Defensive power: The ability to withstand and mitigate damage from opponents and monsters, through physical resilience, armor, or magical protection.\\
- Mobility: The ability to move quickly and efficiently in combat, evading attacks and positioning oneself for strategic advantage.\\
- Tactical skills: The ability to make effective decisions in combat, adapting to changing circumstances and exploiting weaknesses in opponents.\\

Each combat scenario is designed to test a specific combination of these dimensions, with varying levels of difficulty and complexity. Participants are evaluated based on their performance in each scenario, using objective metrics such as damage dealt, damage taken, and time to completion.

\subsubsection{Other Relevant Abilities}

In addition to mana manipulation and combat prowess, there are several other abilities that may be relevant to the assessment of awakened individuals, depending on their specific roles and specializations. These may include:

- Sensory enhancement: The ability to perceive and process sensory information beyond the normal human range, such as enhanced vision, hearing, or smell.\\
- Healing and recovery: The ability to heal oneself or others from injuries or ailments, either through magical means or accelerated natural processes.\\
- Crafting and enchantment: The ability to create and enhance magical items, weapons, or armor, imbuing them with special properties or abilities.\\

For each of these abilities, we propose developing specialized assessment modules that can be administered as needed, based on the specific context and goals of the assessment.

\subsection{Administration and Interpretation Guidelines}

To ensure the reliability and validity of the proposed assessment framework, we have developed a set of guidelines for administering and interpreting the standardized measures. These guidelines cover issues such as:

- Standardization of testing conditions: All assessments should be conducted in a controlled and consistent environment, with standardized equipment and procedures.\\
- Training of assessors: Assessors should be trained in the proper administration and scoring of the measures, to ensure consistency and accuracy.\\
- Interpretation of results: The results of the assessments should be interpreted in the context of established norms and benchmarks, based on data from a representative sample of awakened individuals.\\
- Use of multiple measures: Whenever possible, decisions about an individual's abilities and potential should be based on multiple measures and sources of information, rather than relying on a single assessment.\\

\subsection{Revised F-S Scale}

Based on the results of the standardized assessments, we propose a revised version of the F-S scale that provides a more accurate and reliable classification of awakened individuals and dungeons. The revised scale maintains the basic structure of the original F-S scale, with ranks ranging from F to S, but incorporates the following changes:

- Quantitative criteria: Each rank is defined by specific quantitative criteria, based on the results of the standardized assessments. For example, an A-rank awakened individual might be defined as someone who scores in the top 10\% on both mana manipulation and combat prowess measures.\\
- Multidimensional profiles: In addition to an overall rank, each individual or dungeon is assigned a multidimensional profile that summarizes their specific strengths and weaknesses across different abilities and dimensions.\\
- Adaptive thresholds: The thresholds for each rank are adaptively updated based on new data and research, to ensure that the scale remains valid and relevant over time.\\

Table \ref{fig:scale} presents the revised F-S scale, which incorporates the quantitative criteria derived from the standardized assessment measures. The scale maintains the original rank structure, from F to S, but defines each rank based on specific score ranges across the key dimensions of mana manipulation, combat prowess, and tactical skills.

\begin{table}[ht]
\centering
\begin{tabular}{ccccc}
\hline
\textbf{Rank} & \textbf{Mana Manipulation} & \textbf{Combat Prowess} & \textbf{Tactical Skills} & \textbf{Overall Score Range} \\
\hline
F & 0-20 & 0-20 & 0-20 & 0-60 \\
E & 21-40 & 21-40 & 21-40 & 61-120 \\
D & 41-60 & 41-60 & 41-60 & 121-180 \\
C & 61-80 & 61-80 & 61-80 & 181-240 \\
B & 81-90 & 81-90 & 81-90 & 241-270 \\
A & 91-95 & 91-95 & 91-95 & 271-285 \\
S & 96-100 & 96-100 & 96-100 & 286-300 \\
\hline
\end{tabular}
\caption{The revised F-S scale, with quantitative criteria for each rank based on the standardized assessment measures. Each dimension is scored on a scale from 0 to 100, and the overall score is the sum of the three dimension scores.}
\label{fig:scale}
\end{table}




\subsection{Validation Studies}

To establish the reliability and validity of the proposed assessment framework, we conducted a series of validation studies with a sample of awakened individuals and dungeons. The validation process involved the following steps:

\subsubsection{Reliability Analysis}

We first examined the internal consistency and test-retest reliability of each standardized measure. Internal consistency was evaluated using Cronbach's alpha, with values above 0.7 considered acceptable \citep{Nunnally1978}. Test-retest reliability was evaluated by administering the measures to a subsample of participants at two time points, separated by a two-week interval. Intraclass correlation coefficients (ICCs) were calculated to assess the agreement between the two sets of scores, with values above 0.6 considered acceptable \citep{Cicchetti1994}.

\subsubsection{Construct Validity}

To evaluate the construct validity of the measures, we examined their correlations with existing measures of related constructs, such as physical fitness, cognitive abilities, and personality traits. We hypothesized that the mana manipulation and combat prowess measures would show moderate to strong correlations with measures of physical fitness and combat skills, while showing weaker correlations with measures of cognitive abilities and personality traits.

We also conducted exploratory factor analyses to examine the underlying structure of the measures and identify any potential subcomponents or dimensions. Factors were extracted using principal component analysis, and rotated using the varimax method to enhance interpretability.

\subsubsection{Criterion Validity}

To assess the criterion validity of the measures, we examined their ability to predict relevant outcomes, such as performance in real-world combat scenarios and success in dungeon exploration missions. We collected data on these outcomes from a subsample of participants who had completed the standardized assessments, and conducted regression analyses to examine the predictive power of the measures.

We also compared the scores of experienced and novice hunters, as well as individuals from different ranks and specializations, to evaluate the discriminant validity of the measures. We hypothesized that experienced hunters and higher-ranked individuals would score significantly higher on the relevant measures compared to novice hunters and lower-ranked individuals.

\subsubsection{Results}

The results of the validation studies provided strong support for the reliability and validity of the proposed assessment framework. The internal consistency of the measures was high, with Cronbach's alpha values ranging from 0.78 to 0.92. Test-retest reliability was also adequate, with ICC values ranging from 0.65 to 0.84.

The construct validity of the measures was supported by their correlations with related constructs. As expected, the mana manipulation and combat prowess measures showed moderate to strong correlations with measures of physical fitness and combat skills (r = 0.45 to 0.72), while showing weaker correlations with measures of cognitive abilities and personality traits (r = 0.12 to 0.36). The exploratory factor analyses identified three main factors underlying the measures: mana manipulation, combat prowess, and tactical skills. These factors accounted for 68\% of the total variance in the measures.

The criterion validity of the measures was also supported by their predictive power for relevant outcomes. The mana manipulation and combat prowess measures were significant predictors of performance in real-world combat scenarios (β = 0.36 to 0.58, p < 0.01) and success in dungeon exploration missions (β = 0.42 to 0.67, p < 0.01). The measures also discriminated between experienced and novice hunters, as well as between individuals from different ranks and specializations (Cohen's d = 0.85 to 1.42).

Table \ref{tab:validation} summarizes the key results of the validation studies, providing evidence for the reliability, construct validity, and criterion validity of the proposed assessment framework.

\begin{table}[ht]
\centering
\begin{tabular}{llc}
\hline
Validation Aspect & Measure & Result \\
\hline
\multirow{2}{*}{Reliability} & Internal consistency (Cronbach's alpha) & 0.78 - 0.92 \\
& Test-retest reliability (ICC) & 0.65 - 0.84 \\
\hline
\multirow{2}{*}{Construct validity} & Correlations with related constructs & 0.12 - 0.72 \\
& Variance explained by extracted factors & 68\% \\
\hline
\multirow{2}{*}{Criterion validity} & Predictive power for combat performance (β) & 0.36 - 0.58 \\
& Predictive power for mission success (β) & 0.42 - 0.67 \\
& Discrimination between hunter groups (Cohen's d) & 0.85 - 1.42 \\
\hline
\end{tabular}
\caption{Summary of key results from the validation studies, providing evidence for the reliability, construct validity, and criterion validity of the proposed assessment framework.}
\label{tab:validation}
\end{table}

Overall, these results suggest that the proposed assessment framework is a reliable and valid tool for evaluating the mana manipulation, combat prowess, and other relevant abilities of awakened individuals. The framework can be used to inform decisions about training, team composition, and resource allocation in the context of dungeon exploration and monster combat.

\subsection{Limitations and Future Directions}

Despite the promising results of the validation studies, there are several limitations and areas for future research that should be noted. First, the sample size for the validation studies was relatively small, and may not be fully representative of the broader population of awakened individuals. Future studies should seek to replicate and extend these findings with larger and more diverse samples.

Second, the proposed assessment framework focuses primarily on individual-level abilities and characteristics, and does not fully account for the social and contextual factors that may influence performance in real-world settings. Future research should explore how factors such as team dynamics, leadership, and organizational culture may interact with individual abilities to shape outcomes in dungeon exploration and monster combat.

Third, the proposed framework is based on the current state of knowledge about mana and awakened abilities, which is still limited and evolving. As new discoveries are made and new types of abilities emerge, the framework may need to be updated and refined to maintain its relevance and validity.

Finally, the ethical and societal implications of using standardized assessments for the evaluation and selection of awakened individuals should be carefully considered. While such assessments can provide valuable information and support evidence-based decision making, they also have the potential to reinforce existing inequalities and create new forms of discrimination. Researchers and practitioners should work to ensure that the use of these assessments is fair, transparent, and aligned with broader goals of social justice and inclusivity.

\section{Discussion}

The proposed assessment framework for mana and awakened abilities represents a significant step forward in our understanding and evaluation of these phenomena. By incorporating psychometric principles and standardized measures, the framework offers a more reliable and valid approach to assessing the abilities and potential of awakened individuals, and informing decisions about their training, deployment, and support.

One of the key strengths of the framework is its grounding in empirical data and validation studies. The results of these studies provide strong evidence for the reliability and validity of the proposed measures, and suggest that they can be useful tools for predicting performance and success in real-world contexts. This is particularly important given the high stakes and risks associated with dungeon exploration and monster combat, where accurate assessment and selection of personnel can be a matter of life and death.

Another important contribution of the framework is its multidimensional approach to assessing abilities and potential. By evaluating individuals across multiple dimensions, such as mana manipulation, combat prowess, and tactical skills, the framework provides a more comprehensive and nuanced picture of their strengths and weaknesses. This can help to inform more targeted and effective training and support interventions, as well as more strategic decisions about team composition and resource allocation.

However, it is important to recognize that the proposed framework is not a panacea, and that there are significant challenges and limitations that must be addressed. One of the most pressing challenges is the need for ongoing validation and refinement of the measures, as new types of abilities and challenges emerge in the rapidly evolving landscape of dungeon exploration and monster combat. This will require a sustained commitment to research and data collection, as well as a willingness to adapt and update the framework as needed.

Another key challenge is the need to consider the broader social and ethical implications of using standardized assessments for the evaluation and selection of awakened individuals. While such assessments can provide valuable information and support evidence-based decision making, they also have the potential to reinforce existing inequalities and create new forms of discrimination. For example, if certain demographic groups are systematically underrepresented among high-performing awakened individuals, the use of standardized assessments could lead to their further marginalization and exclusion.

To mitigate these risks, it is essential that the development and use of the proposed framework be guided by principles of fairness, transparency, and inclusivity. This may involve efforts to ensure that the measures are culturally sensitive and do not disadvantage certain groups, as well as initiatives to promote diversity and equity in the recruitment, training, and support of awakened individuals.

Ultimately, the success of the proposed assessment framework will depend on its ability to balance the need for reliable and valid measurement with the imperative to promote social justice and human wellbeing. This will require ongoing collaboration and dialogue between researchers, practitioners, and stakeholders, as well as a commitment to using the framework in ways that align with broader goals of individual and collective flourishing.

In conclusion, the proposed assessment framework for mana and awakened abilities represents an important step forward in our understanding and evaluation of these phenomena. While there are significant challenges and limitations that must be addressed, the framework offers a promising approach to informing evidence-based decision making and supporting the effective development and deployment of awakened individuals in the service of humanity. As we continue to navigate the complex and uncertain landscape of the post-catastrophe world, such tools and frameworks will be essential for ensuring that we are able to harness the power of mana and awakened abilities in ways that promote the greater good.

\section{Conclusion}

The emergence of dungeons and awakened individuals in the wake of the 2025 catastrophe has presented humanity with unprecedented challenges and opportunities. On one hand, the dungeons represent a new frontier for exploration, discovery, and resource acquisition, offering the potential to unlock new knowledge, technologies, and forms of power. On the other hand, the dangers posed by the monsters and hazards within these dungeons, as well as the risks of misuse or abuse of awakened abilities, underscore the need for careful and responsible management of these phenomena.

The proposed assessment framework for mana and awakened abilities seeks to contribute to this effort by providing a reliable and valid tool for evaluating the abilities and potential of awakened individuals. By incorporating psychometric principles and standardized measures, the framework offers a more systematic and evidence-based approach to assessment, which can inform decisions about training, deployment, and support in a variety of contexts.

However, it is important to recognize that the framework is not a silver bullet, and that its effectiveness will depend on how it is used and integrated into broader systems and practices. To maximize its potential benefits and minimize its potential harms, the framework must be implemented in ways that are consistent with principles of fairness, transparency, and accountability, and that prioritize the wellbeing and flourishing of both awakened individuals and society as a whole.

This will require ongoing collaboration and dialogue between researchers, practitioners, policymakers, and other stakeholders, as well as a commitment to using the framework in ways that align with broader goals of social justice, human rights, and sustainability. It will also require a willingness to adapt and refine the framework as new evidence and insights emerge, and to remain open to alternative approaches and perspectives.

Ultimately, the success of the proposed assessment framework will be measured not only by its technical merits, but also by its ability to contribute to the development of a more just, equitable, and thriving post-catastrophe world. By providing a foundation for evidence-based decision making and support, the framework has the potential to play a valuable role in this process, but it is only one piece of a much larger puzzle.

As we move forward, it will be essential to continue to invest in research, innovation, and collaboration aimed at understanding and managing the complex phenomena of dungeons and awakened abilities. This will require not only technical expertise and resources, but also a deep commitment to ethical and social responsibility, and a recognition of the inherent dignity and worth of all human beings.

In the end, the true test of our success will be the kind of world we are able to create in the aftermath of the catastrophe – a world that is not only safe, prosperous, and technologically advanced, but also just, compassionate, and fully human. The proposed assessment framework for mana and awakened abilities is a step in this direction, but it is only the beginning of a much longer and more challenging journey.

\bibliographystyle{apalike}
\bibliography{references}

\end{document}
