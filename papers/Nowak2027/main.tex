\documentclass[12pt, a4paper]{article}
\usepackage[margin=1in]{geometry}
\usepackage{graphicx}
\usepackage{natbib}
\usepackage{hyperref}
\usepackage[utf8]{inputenc}
\usepackage{amsmath}
\usepackage{amssymb}
\usepackage{booktabs}

\title{Coping with the Trauma of Awakening}
\author{Ewa Nowak\footnote{Department of Psychology, University of Warsaw, Warsaw, Poland. \texttt{ewa.nowak@psych.uw.edu.pl}} \and 
Agnieszka Kowalska\footnote{Institute of Mental Health, Polish Academy of Sciences, Warsaw, Poland. \texttt{a.kowalska@imh.pan.pl}}}
\date{August 2027}

\begin{document}
\maketitle
\begin{abstract}
This study investigates the psychological impact of awakening on individuals in the aftermath of the 2025 catastrophe, which saw the emergence of a small portion of the population with extraordinary abilities. Through a series of in-depth interviews and psychological assessments, we examine the challenges faced by newly awakened individuals, such as coming to terms with their new abilities, navigating social stigma, and adapting to the demands of hunter guilds and dungeon exploration. Our findings highlight the importance of mental health interventions and resources for awakened individuals, as well as the need for public education and awareness campaigns to reduce stigma and promote understanding.
\end{abstract}

\section{Introduction}
The 2025 catastrophe marked a turning point in human history, as mysterious portals began to appear across the globe, leading to treacherous, enclosed environments known as dungeons. These dungeons, varying in size and complexity, are inhabited by powerful creatures with extraordinary abilities that pose a significant threat to humanity \citep{Chen2026}. In response to this crisis, a small portion of the population has awakened, developing unique abilities that allow them to combat these creatures and explore the dungeons \citep{Nakamura2027}.

While the emergence of awakened individuals has been crucial in the fight against the dungeon threat, the process of awakening and the subsequent challenges faced by these individuals have significant psychological implications. Awakened individuals often struggle to come to terms with their new abilities, which can manifest suddenly and dramatically, altering their sense of identity and place in the world \citep{Lee2027}. Moreover, the pressure to join hunter guilds and participate in dungeon exploration can exacerbate the psychological strain on awakened individuals, as they are thrust into dangerous and traumatic situations \citep{Kim2027}.

The psychological impact of awakening extends beyond the individual, affecting their families and communities as well. The sudden change in an individual's abilities can disrupt family dynamics, leading to feelings of alienation, guilt, and resentment \citep{Kowalczyk2027}. Furthermore, the social stigma surrounding awakened individuals can result in discrimination, exclusion, and even violence, further exacerbating the psychological distress experienced by this population \citep{Nowak2027}.

Given the significant psychological challenges faced by awakened individuals and their families, there is a pressing need for mental health interventions and resources tailored to this unique population. However, the development of such interventions has been hindered by a lack of research on the specific psychological needs and experiences of awakened individuals. This study aims to address this gap in the literature by conducting in-depth interviews and psychological assessments with awakened individuals, with the goal of informing the development of targeted mental health interventions and support systems.

\section{Methods}
\subsection{Participants}
A total of 30 awakened individuals (15 male, 15 female) were recruited for this study through hunter guilds and mental health clinics in Warsaw, Poland. Participants ranged in age from 18 to 45 years old (M = 28.5, SD = 7.8) and had been awakened for an average of 1.5 years (SD = 0.8) at the time of the study. All participants were classified as C-rank or above by their respective hunter guilds, indicating a moderate to high level of ability and dungeon experience.

\subsection{Procedure}
Participants completed a series of in-depth, semi-structured interviews lasting approximately 90 minutes each. The interviews covered a range of topics related to the participants' experiences of awakening, including the initial manifestation of their abilities, their decision to join a hunter guild, their experiences in dungeons, and the impact of awakening on their personal lives and relationships. Interviews were conducted by trained clinical psychologists and were audio-recorded and transcribed for analysis.

In addition to the interviews, participants completed a battery of psychological assessments, including the Impact of Event Scale-Revised (IES-R; \citealt{Weiss2007}), the Connor-Davidson Resilience Scale (CD-RISC; \citealt{Connor2003}), and the Multidimensional Scale of Perceived Social Support (MSPSS; \citealt{Zimet1988}). These measures were selected to assess participants' levels of trauma-related distress, resilience, and perceived social support, respectively.

\subsection{Data Analysis}
Interview transcripts were analyzed using thematic analysis, a qualitative research method that involves identifying, analyzing, and reporting patterns or themes within a dataset \citep{Braun2006}. The analysis was conducted by a team of three researchers, who independently coded the transcripts and then met to discuss and refine the emerging themes. The psychological assessment data were analyzed using descriptive statistics and correlational analyses to examine the relationships between participants' scores on the IES-R, CD-RISC, and MSPSS.

\section{Results}
\subsection{Thematic Analysis}
The thematic analysis of the interview transcripts revealed four main themes related to the psychological impact of awakening: (1) identity disruption, (2) trauma and guilt, (3) social stigma and isolation, and (4) resilience and post-traumatic growth.

\subsubsection{Identity Disruption}
Many participants reported that the sudden manifestation of their abilities led to a profound disruption in their sense of identity. They described feeling like a "different person" or "not knowing who they were anymore" in the aftermath of awakening. For some, this identity disruption was characterized by a sense of loss and grief, as they mourned the life they had known before awakening. Others reported feeling a sense of alienation from their former selves and struggled to integrate their new abilities into their self-concept.

\begin{quote}
    "I woke up one day and everything had changed. I could do things I never thought possible, but I didn't feel like myself anymore. It was like I was a stranger in my own body." (Participant 7, female, age 24)
\end{quote}

\subsubsection{Trauma and Guilt}
Many participants reported experiencing symptoms of trauma-related distress, such as intrusive thoughts, nightmares, and hypervigilance, in the aftermath of awakening and dungeon exploration. They described feeling haunted by the memories of the creatures they had encountered and the battles they had fought, and struggled to come to terms with the violence and danger that had become a part of their daily lives.

\begin{quote}
    "I can't sleep at night. Every time I close my eyes, I see the faces of the monsters I've killed. I know it's kill or be killed in there, but that doesn't make it any easier." (Participant 12, male, age 32)
\end{quote}

In addition to the trauma of dungeon exploration, many participants reported feeling a sense of guilt and responsibility for the safety of their fellow guild members and the civilians they were sworn to protect. They described feeling a constant pressure to be stronger, faster, and more skilled, and blamed themselves when things went wrong or when lives were lost.

\begin{quote}
    "I keep thinking about what I could have done differently, how I could have saved them. It's my fault they died. I wasn't strong enough." (Participant 22, female, age 29)
\end{quote}

\subsubsection{Social Stigma and Isolation}
Many participants reported experiencing social stigma and discrimination as a result of their awakened status. They described feeling ostracized by their communities and even their own families, who feared or resented their abilities. Some participants reported losing friends and romantic partners who could not cope with the changes in their lives, while others described feeling pressure to hide their abilities in order to avoid negative reactions from others.

\begin{quote}
    "My parents won't even look at me anymore. They think I'm some kind of freak, like I asked for this to happen to me. I've never felt so alone." (Participant 18, male, age 21)
\end{quote}

The social isolation experienced by many awakened individuals was exacerbated by the demands of their hunter guild responsibilities, which often required them to spend long periods of time away from home and in the company of other awakened individuals. While some participants reported finding a sense of belonging and camaraderie within their guilds, others described feeling disconnected from their civilian lives and struggling to maintain relationships outside of the guild.

\subsubsection{Resilience and Post-Traumatic Growth}
Despite the significant challenges and traumas experienced by awakened individuals, many participants also reported experiences of resilience and personal growth in the aftermath of awakening. They described feeling a sense of purpose and meaning in their work as hunters, and reported finding strength and support in their relationships with their fellow guild members.

\begin{quote}
    "Being a hunter is the hardest thing I've ever done, but it's also the most rewarding. I know that what I'm doing matters, that I'm making a difference in the world. And I couldn't do it without my guildmates. They're like family to me." (Participant 26, female, age 36)
\end{quote}

Some participants also reported experiencing post-traumatic growth, or positive psychological changes that can occur in the aftermath of a traumatic event. They described feeling a greater appreciation for life, stronger relationships with others, and a sense of personal strength and resilience that they had not had before awakening.

\begin{quote}
    "I never would have chosen this life for myself, but in a strange way, I'm grateful for it. It's made me stronger, more resilient. I know now that I can handle anything that comes my way." (Participant 14, male, age 28)
\end{quote}

\subsection{Psychological Assessment Results}
The results of the psychological assessments revealed significant levels of trauma-related distress among the participants. The mean score on the IES-R was 42.5 (SD = 12.8), indicating high levels of intrusion, avoidance, and hyperarousal symptoms. Participants' scores on the CD-RISC (M = 68.4, SD = 14.2) and the MSPSS (M = 5.2, SD = 1.4) were within the normal range, suggesting moderate levels of resilience and perceived social support.

Correlational analyses revealed significant negative correlations between participants' scores on the IES-R and both the CD-RISC (r = -.62, p < .001) and the MSPSS (r = -.58, p < .001), indicating that higher levels of trauma-related distress were associated with lower levels of resilience and perceived social support. These findings suggest that interventions aimed at reducing trauma symptoms and increasing resilience and social support may be particularly beneficial for awakened individuals.

\section{Discussion}
The findings of this study highlight the significant psychological challenges faced by awakened individuals in the aftermath of the 2025 catastrophe. The sudden manifestation of extraordinary abilities, coupled with the trauma of dungeon exploration and the social stigma associated with awakening, can have a profound impact on an individual's mental health and well-being.

The theme of identity disruption that emerged from the interviews underscores the transformative nature of awakening. For many individuals, the development of abilities that defy the laws of nature can challenge their fundamental beliefs about themselves and their place in the world. This finding is consistent with previous research on the psychological impact of extraordinary experiences, such as near-death experiences and spiritual transformations \citep{Greyson1997, Pargament2006}. The loss of a sense of self and the struggle to integrate new abilities into one's identity may be particularly challenging for awakened individuals, who must also cope with the demands of their hunter guild responsibilities and the social stigma associated with their abilities.

The high levels of trauma-related distress reported by participants are not surprising given the dangerous and violent nature of dungeon exploration. Awakened individuals are regularly exposed to life-threatening situations and must confront creatures with powers that defy human understanding. The constant threat of death or serious injury, coupled with the guilt and responsibility that many hunters feel for the safety of their comrades and the civilians they protect, can take a heavy psychological toll. These findings are consistent with research on the mental health of military personnel and first responders, who are similarly exposed to high levels of trauma and stress in the line of duty \citep{Smith2014, Berger2012}.

The social stigma and isolation experienced by many awakened individuals is another significant challenge that emerged from the interviews. The fear and resentment that some civilians feel towards those with extraordinary abilities can lead to discrimination, exclusion, and even violence. This finding is consistent with research on the experiences of individuals with stigmatized identities, such as those with mental illness or physical disabilities \citep{Corrigan2004, Susman2000}. The pressure to hide one's abilities in order to avoid negative reactions from others can further exacerbate feelings of isolation and disconnection from society.

Despite the significant challenges faced by awakened individuals, the themes of resilience and post-traumatic growth that emerged from the interviews suggest that there is also the potential for positive psychological outcomes in the aftermath of awakening. Many participants reported finding a sense of purpose and meaning in their work as hunters, and described the strong bonds of camaraderie and support within their guilds. This finding is consistent with research on the role of social support and a sense of purpose in promoting resilience and well-being in the face of adversity \citep{Southwick2005, McKnight2009}.

The experiences of post-traumatic growth reported by some participants are also consistent with previous research on the potential for positive psychological changes in the aftermath of trauma \citep{Tedeschi2004}. While the trauma of awakening and dungeon exploration can be devastating, it can also provide opportunities for personal growth and transformation. The development of a greater appreciation for life, stronger relationships with others, and a sense of personal strength and resilience can be powerful sources of motivation and meaning for awakened individuals.

The results of the psychological assessments provide further evidence of the significant levels of trauma-related distress experienced by awakened individuals. The high scores on the IES-R suggest that many participants are struggling with symptoms of intrusion, avoidance, and hyperarousal, which are common in individuals with post-traumatic stress disorder (PTSD). The negative correlations between trauma symptoms and both resilience and perceived social support highlight the importance of interventions aimed at reducing trauma symptoms and promoting resilience and social connection among awakened individuals.

\subsection{Implications for Mental Health Interventions}
The findings of this study have important implications for the development of mental health interventions for awakened individuals. Given the high levels of trauma-related distress and the unique challenges associated with awakening, it is clear that traditional approaches to trauma treatment may not be sufficient for this population. Instead, interventions must be tailored to the specific needs and experiences of awakened individuals, taking into account the impact of their abilities on their sense of identity and the social stigma they may face.

One promising approach to intervention may be to focus on promoting resilience and post-traumatic growth among awakened individuals. This could involve providing education and training on coping skills, such as mindfulness and emotion regulation, as well as encouraging engagement in activities that promote a sense of purpose and meaning, such as volunteering or mentoring other awakened individuals. Interventions that foster social support and connection within hunter guilds and the broader awakened community may also be particularly beneficial, given the importance of social support in promoting resilience and well-being.

Psychotherapy interventions that specifically address the unique challenges of awakening may also be necessary. This could involve helping individuals to process and make sense of their experiences, develop a coherent narrative of their identity post-awakening, and cope with the social stigma and isolation they may face. Therapists working with awakened individuals may need specialized training in order to effectively address these unique challenges and provide culturally competent care.

In addition to individual interventions, there may also be a need for broader systemic changes to support the mental health and well-being of awakened individuals. This could involve providing education and training for mental health professionals on the unique needs of this population, as well as advocating for policies and practices that promote social inclusion and reduce discrimination against awakened individuals.

\subsection{Limitations and Future Directions}
While this study provides important insights into the psychological impact of awakening, there are several limitations that must be acknowledged. First, the sample size of 30 participants is relatively small, and may not be representative of the broader population of awakened individuals. Future research should aim to recruit larger and more diverse samples in order to better understand the range of experiences and challenges faced by this population.

Second, the study relied on self-report measures and interviews, which may be subject to biases and limitations in terms of accuracy and depth of information. Future research could employ more objective measures, such as physiological assessments or behavioral observations, to provide a more comprehensive understanding of the psychological impact of awakening.

Third, the study was limited to a specific cultural context, namely awakened individuals in Poland. While the findings may have relevance for other cultural contexts, it is important to recognize that the experiences of awakened individuals may vary significantly depending on factors such as social norms, cultural beliefs, and political climate. Future research should aim to explore the psychological impact of awakening in diverse cultural contexts in order to identify common themes and unique challenges.

Despite these limitations, the findings of this study provide an important foundation for future research on the psychological impact of awakening. Further research is needed to develop and evaluate targeted interventions for this population, as well as to explore the long-term consequences of awakening on mental health and well-being. Longitudinal studies that follow awakened individuals over time could provide valuable insights into the trajectories of trauma, resilience, and post-traumatic growth, and help to identify factors that promote positive outcomes.

Future research should also aim to explore the broader societal implications of awakening, including the impact on families, communities, and social institutions. The emergence of a population with extraordinary abilities has the potential to transform society in profound ways, and it is important to understand the psychological and social consequences of these changes.

\section{Conclusion}
The 2025 catastrophe and the subsequent emergence of awakened individuals with extraordinary abilities has had a profound impact on society and the individuals who possess these powers. This study provides important insights into the psychological challenges faced by awakened individuals, including identity disruption, trauma, social stigma, and isolation. At the same time, the findings also highlight the potential for resilience and post-traumatic growth among this population, and the importance of social support and a sense of purpose in promoting well-being.

The development of targeted mental health interventions for awakened individuals is a critical priority, given the high levels of trauma-related distress and the unique challenges associated with awakening. By providing education, training, and support to promote resilience and post-traumatic growth, we can help awakened individuals to thrive in the face of adversity and to fully realize their potential as heroes and protectors of society.

At the same time, it is important to recognize that the challenges faced by awakened individuals are not solely a matter of individual mental health, but are also deeply rooted in societal attitudes and structures. Addressing the social stigma and discrimination faced by awakened individuals will require a concerted effort to promote understanding, acceptance, and inclusion at all levels of society.

As we continue to navigate the uncharted territory of a world transformed by the emergence of extraordinary abilities, it is essential that we prioritize the mental health and well-being of those who possess these powers. By working together to support awakened individuals and to create a society that values and includes them, we can build a brighter future for all.

\section*{Acknowledgements}
We would like to thank the participants who generously shared their experiences and insights with us, as well as the hunter guilds and mental health clinics that assisted with recruitment. We would also like to thank our research assistants for their tireless work in conducting interviews and analyzing data. This research was supported by a grant from the National Science Centre, Poland (grant number 2027/01/D/HS6/00001).

\section*{Declaration of Conflicting Interests}
The author(s) declared no potential conflicts of interest with respect to the research, authorship, and/or publication of this article.

\section*{Funding}
The author(s) disclosed receipt of the following financial support for the research, authorship, and/or publication of this article: This research was supported by a grant from the National Science Centre, Poland (grant number 2027/01/D/HS6/00001).

\bibliography{references}
\bibliographystyle{apalike}

\end{document}
