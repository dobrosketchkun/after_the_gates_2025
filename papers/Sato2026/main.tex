\documentclass[12pt, a4paper]{article}
\usepackage[utf8]{inputenc}
\usepackage[english]{babel}
\usepackage{graphicx}
\usepackage{subcaption}
\usepackage{amsmath}
\usepackage{amssymb}
\usepackage{hyperref}
\usepackage[left=2.5cm, right=2.5cm, top=3cm, bottom=3cm]{geometry}

\title{The Organizational Structures of Hunter Guilds in Japan}
\author{
    Ryusei Sato\textsuperscript{1},
    Naoki Takahashi\textsuperscript{2},
    Hiroshi Watanabe\textsuperscript{1} \\
    \textsuperscript{1}Department of Sociology, University of Tokyo, Tokyo, Japan \\
    \textsuperscript{2}School of Business Administration, Kyoto University, Kyoto, Japan
}
\date{2026}

\begin{document}

\maketitle

\begin{abstract}
This study investigates the diverse organizational structures of hunter guilds that have emerged in Japan following the 2025 catastrophe and the subsequent appearance of portals leading to dangerous, resource-rich environments known as dungeons. Drawing on a nationwide survey and in-depth interviews with guild leaders and members, the authors identify and analyze four distinct types of guild structures: (1) Hierarchical Guilds: characterized by a clear chain of command, centralized decision-making, and specialized roles for members; (2) Networked Guilds: consisting of loosely connected, independent hunters who collaborate on specific missions; (3) Cooperative Guilds: featuring shared ownership and democratic decision-making among members; and (4) Hybrid Guilds: combining elements of hierarchical, networked, and cooperative structures. The authors argue that the choice of organizational structure is influenced by factors such as guild size, the complexity of dungeons explored, the degree of specialization among hunters, and the level of trust and cooperation within the guild. The study also examines the advantages and disadvantages associated with each type of structure, highlighting the trade-offs between efficiency, flexibility, and member autonomy. The findings contribute to a deeper understanding of the organizational dynamics of hunter guilds and provide insights into how these structures adapt to the challenges and opportunities of the post-catastrophe world.
\end{abstract}

\section{Introduction}
The emergence of mysterious portals in 2025, known as the "Portal Catastrophe," has dramatically altered the global landscape, presenting humanity with unprecedented challenges and opportunities. These portals lead to diverse, enclosed environments called dungeons, which are filled with dangerous creatures and valuable resources \cite{Nakamura2025}. In response to this new reality, a small portion of the population has developed extraordinary abilities, becoming "awakened" individuals with the potential to explore and conquer these dungeons \cite{Yamamoto2026}.

The rise of awakened individuals has given birth to a new type of organization: the hunter guild. Hunter guilds are groups of awakened individuals who work together to explore dungeons, combat the creatures within, and extract valuable resources \cite{Sato2025}. These guilds have quickly become essential actors in the post-catastrophe world, shaping economic, social, and political dynamics at local, national, and global levels \cite{Watanabe2026}.

Despite their importance, little is known about the organizational structures of hunter guilds and how these structures influence their effectiveness, resilience, and adaptability in the face of the challenges posed by dungeons. This study aims to address this gap by investigating the diverse organizational structures of hunter guilds in Japan, one of the countries most affected by the Portal Catastrophe \cite{Mori2025}.

\subsection{Research Questions}
The study seeks to answer the following research questions:
\begin{enumerate}
    \item What are the main types of organizational structures adopted by hunter guilds in Japan?
    \item What factors influence the choice of organizational structure among hunter guilds?
    \item What are the advantages and disadvantages associated with each type of organizational structure?
    \item How do different organizational structures affect the performance and adaptability of hunter guilds in the post-catastrophe world?
\end{enumerate}

\subsection{Significance of the Study}
Understanding the organizational dynamics of hunter guilds is crucial for several reasons. First, it can help identify best practices and strategies for effective dungeon exploration and resource management, contributing to the success and survival of hunter guilds \cite{Nakamura2025}. Second, it can inform policymakers and stakeholders in their efforts to regulate, support, or collaborate with hunter guilds, ensuring that these organizations operate in a manner that benefits society as a whole \cite{Yoshida2025}. Third, it can provide insights into the broader implications of the Portal Catastrophe for organizational theory and practice, shedding light on how organizations adapt to radical environmental changes and how new forms of organizing emerge in response to novel challenges \cite{Takahashi2026}.

\section{Literature Review}
\subsection{The Portal Catastrophe and the Emergence of Dungeons}
The Portal Catastrophe of 2025 marked a turning point in human history, as portals began to appear worldwide, connecting Earth to previously unknown, enclosed environments called dungeons \cite{Nakamura2025}. These dungeons vary in size, structure, and content, ranging from small, cave-like spaces to vast, complex landscapes that defy conventional laws of physics and biology \cite{Yamamoto2025}. Dungeons are inhabited by diverse creatures, many of which possess abilities and characteristics that challenge existing scientific knowledge \cite{Sato2025a}.

The appearance of dungeons has had profound consequences for societies around the globe. On one hand, dungeons pose significant threats to human life and infrastructure, as the creatures within them can be highly aggressive and destructive \cite{Watanabe2026}. On the other hand, dungeons contain valuable resources, such as rare materials, advanced technologies, and medicinal substances, which have the potential to revolutionize various industries and improve human well-being \cite{Mori2026}.

\subsection{The Rise of Awakened Individuals and Hunter Guilds}
In the wake of the Portal Catastrophe, a small portion of the population has developed extraordinary abilities, becoming "awakened" individuals \cite{Yamamoto2026}. These abilities vary widely, ranging from enhanced physical strength and sensory perception to psychokinetic powers and elemental manipulation \cite{Nakamura2026a}. Awakened individuals are classified into ranks based on the strength and rarity of their abilities, with S-rank being the highest and F-rank being the lowest \cite{Sato2025}.

The emergence of awakened individuals has led to the formation of hunter guilds, organizations that specialize in dungeon exploration and resource extraction \cite{Sato2025}. Hunter guilds provide a framework for awakened individuals to collaborate, share knowledge, and develop their abilities while navigating the challenges posed by dungeons \cite{Watanabe2026}. These guilds have quickly become key players in the post-catastrophe world, shaping economic, social, and political dynamics at various scales \cite{Yoshida2025}.

\subsection{Organizational Theory and the Study of Hunter Guilds}
Organizational theory provides a useful lens for understanding the structures and dynamics of hunter guilds. Classical organizational theories, such as Weber's bureaucracy \cite{Weber1947} and Taylor's scientific management \cite{Taylor1911}, emphasize the importance of formal hierarchies, standardized procedures, and specialization in achieving efficiency and effectiveness. These theories have been influential in shaping modern organizations, but their applicability to hunter guilds remains unclear, given the unique challenges and uncertainties posed by dungeons \cite{Takahashi2026}.

More recent organizational theories, such as contingency theory \cite{Lawrence1967} and resource dependence theory \cite{Pfeffer1978}, suggest that organizations must adapt their structures and strategies to fit their environmental contexts. These theories may be more relevant to hunter guilds, as they highlight the importance of flexibility, adaptability, and external relationships in navigating complex and changing environments \cite{Watanabe2026a}.

Network theories of organization, such as social network analysis \cite{Borgatti2009} and actor-network theory \cite{Latour2005}, offer another perspective on hunter guilds. These theories emphasize the role of informal networks, social ties, and non-human actors in shaping organizational outcomes. They may be particularly useful for understanding the collaborative dynamics within and between hunter guilds, as well as the ways in which technology, resources, and creatures influence guild structures and strategies \cite{Sato2026}.

Despite the potential insights offered by organizational theory, empirical research on hunter guilds remains limited. Most studies have focused on the economic and social impacts of dungeons and awakened individuals, rather than the organizational aspects of hunter guilds \cite{Mori2026, Yoshida2025}. This study aims to fill this gap by providing a comprehensive analysis of the organizational structures of hunter guilds in Japan.

\section{Methodology}
\subsection{Research Design}
To investigate the organizational structures of hunter guilds in Japan, we employed a mixed-methods research design combining a nationwide survey and in-depth interviews. The survey aimed to capture the diversity of guild structures and the factors influencing their choice, while the interviews sought to provide a deeper understanding of the advantages, disadvantages, and implications of different structures.

\subsection{Sample and Data Collection}
The survey sample consisted of 500 hunter guilds randomly selected from the official registry maintained by the Japanese government. The survey questionnaire was distributed online and included items on guild characteristics (e.g., size, age, location), organizational structure (e.g., hierarchy, decision-making processes, specialization), and performance (e.g., dungeon clearance rates, resource extraction, member satisfaction). The survey achieved a response rate of 80\%, with 400 guilds completing the questionnaire.

For the interviews, we purposively selected 20 guild leaders and 40 guild members from the survey sample, ensuring variation in terms of guild size, structure, and performance. The interviews were conducted face-to-face or via video conferencing and followed a semi-structured protocol covering topics such as the rationale behind the chosen organizational structure, the challenges and opportunities associated with different structures, and the perceived impact of structure on guild effectiveness and adaptability. The interviews were audio-recorded and transcribed verbatim.

\subsection{Data Analysis}
The survey data were analyzed using descriptive statistics and exploratory factor analysis (EFA) to identify the main types of organizational structures adopted by hunter guilds. The EFA results were used to classify guilds into four categories: hierarchical, networked, cooperative, and hybrid. Multivariate regression analysis was then conducted to examine the relationships between guild characteristics, organizational structure, and performance outcomes.

The interview data were analyzed using thematic analysis \cite{Braun2006}, with a focus on identifying the key advantages, disadvantages, and implications of different organizational structures. The analysis followed an iterative process of coding, categorization, and interpretation, with multiple researchers involved to ensure reliability and validity.

\subsection{Ethical Considerations}
The study was approved by the institutional review board of the University of Tokyo. All participants provided informed consent, and their anonymity and confidentiality were protected throughout the research process. The findings are reported in aggregate form, and any identifying information has been removed from the interview quotes used in the paper.

\section{Results}
\subsection{Types of Organizational Structures}
The EFA results revealed four distinct types of organizational structures adopted by hunter guilds in Japan: hierarchical, networked, cooperative, and hybrid. Table \ref{tab:structures} presents the key characteristics of each structure type.

\begin{table}[h]
\centering
\caption{Key characteristics of hunter guild organizational structures}
\label{tab:structures}
\begin{tabular}{p{2.5cm}p{3.5cm}p{3.5cm}p{3.5cm}}
\hline
\textbf{Structure} & \textbf{Characteristics} & \textbf{Advantages} & \textbf{Disadvantages} \\
\hline
Hierarchical & 
- Clear chain of command\newline
- Centralized decision-making\newline
- Specialized roles and divisions
& 
- Efficiency\newline
- Coordination\newline
- Clear authority
&
- Rigidity\newline
- Lack of creativity\newline
- Potential for power abuse
\\
\hline
Networked & 
- Loosely connected independent hunters\newline
- Collaboration on specific missions\newline
- Fluid and informal structure
&
- Flexibility\newline
- Adaptability\newline
- Diverse skill sets
&
- Coordination challenges\newline
- Accountability issues\newline
- Lack of long-term stability
\\
\hline
Cooperative & 
- Shared ownership and decision-making\newline
- Democratic governance\newline
- Egalitarian values and practices
&
- Inclusivity\newline
- Solidarity\newline
- Member empowerment
&
- Slow decision-making\newline
- Conflict resolution difficulties\newline
- Lack of external legitimacy
\\
\hline
Hybrid & 
- Combines elements of hierarchical, networked, and cooperative structures\newline
- Flexible and adaptable\newline
- Balances efficiency and autonomy
&
- Balance between structure and flexibility\newline
- Efficiency and autonomy\newline
- Leverages strengths of different approaches
&
- Maintaining coherence and consistency\newline
- Member buy-in challenges\newline
- Requires extensive communication and negotiation
\\
\hline
\end{tabular}
\end{table}

Hierarchical guilds (32\% of the sample) are characterized by a clear chain of command, centralized decision-making, and specialized roles and divisions. These guilds tend to be larger and more established, with a focus on efficiency and standardization. Networked guilds (28\%) consist of loosely connected independent hunters who collaborate on specific missions. These guilds are more fluid and informal, with a focus on flexibility and autonomy. Cooperative guilds (25\%) feature shared ownership and decision-making, democratic governance, and egalitarian values and practices. These guilds prioritize member participation and well-being. Hybrid guilds (15\%) combine elements of hierarchical, networked, and cooperative structures, seeking to balance efficiency, flexibility, and member autonomy.

\subsection{Factors Influencing the Choice of Organizational Structure}
The regression analysis results indicated that several factors significantly influence the choice of organizational structure among hunter guilds. Guild size was positively associated with the adoption of hierarchical structures ($\beta = 0.35$, $p < 0.01$), suggesting that larger guilds tend to favor more formalized and centralized arrangements. Dungeon complexity was positively associated with the adoption of networked structures ($\beta = 0.28$, $p < 0.01$), indicating that guilds exploring more challenging and varied dungeons tend to rely on more flexible and collaborative arrangements. The level of specialization among hunters was positively associated with the adoption of cooperative structures ($\beta = 0.24$, $p < 0.01$), suggesting that guilds with more diverse and complementary skill sets tend to favor more egalitarian and participatory arrangements. Trust and cooperation within the guild were positively associated with the adoption of hybrid structures ($\beta = 0.31$, $p < 0.01$), indicating that guilds with strong social bonds and shared values tend to combine elements of different structures to balance competing demands.

\subsection{Advantages and Disadvantages of Different Organizational Structures}
The interview data revealed several advantages and disadvantages associated with each type of organizational structure. Hierarchical guilds were praised for their efficiency, coordination, and clear lines of authority, but criticized for their rigidity, lack of creativity, and potential for power abuse. As one guild leader noted, "Our hierarchical structure allows us to make quick decisions and mobilize resources effectively, but it can also stifle innovation and breed resentment among lower-ranked members."

Networked guilds were commended for their flexibility, adaptability, and ability to tap into diverse skill sets, but challenged by issues of coordination, accountability, and long-term stability. One guild member explained, "Working in a networked guild gives me a lot of freedom and variety, but it can also be chaotic and unpredictable. It's hard to build trust and cohesion when you're constantly switching partners and projects."

Cooperative guilds were appreciated for their inclusivity, solidarity, and member empowerment, but faced difficulties in decision-making, conflict resolution, and external legitimacy. A guild leader shared, "Our cooperative structure fosters a strong sense of community and shared purpose, but it can also lead to endless debates and a lack of decisive action. Some clients and partners find it hard to take us seriously as a professional organization."

Hybrid guilds were valued for their ability to balance structure and flexibility, efficiency and autonomy, but confronted challenges in maintaining coherence, consistency, and member buy-in. One guild member reflected, "Our hybrid structure allows us to adapt to different situations and leverage the strengths of different approaches, but it can also be confusing and frustrating at times. It requires a lot of communication and negotiation to keep everyone on the same page."

\subsection{Organizational Structures and Guild Performance}
The regression analysis results indicated that organizational structure has a significant impact on various performance outcomes for hunter guilds. Hierarchical structures were positively associated with dungeon clearance rates ($\beta = 0.29$, $p < 0.01$) and resource extraction volumes ($\beta = 0.33$, $p < 0.01$), but negatively associated with member satisfaction ($\beta = -0.22$, $p < 0.01$). These findings suggest that while hierarchical guilds may be more effective in achieving task-related goals, they may struggle to maintain member morale and well-being.

Networked structures were positively associated with innovation and adaptation ($\beta = 0.27$, $p < 0.01$), but negatively associated with coordination and efficiency ($\beta = -0.25$, $p < 0.01$). These results indicate that while networked guilds may be more responsive to changing circumstances and able to generate novel solutions, they may face challenges in executing complex plans and maintaining smooth operations.

Cooperative structures were positively associated with member satisfaction ($\beta = 0.36$, $p < 0.01$) and social impact ($\beta = 0.31$, $p < 0.01$), but negatively associated with financial performance ($\beta = -0.20$, $p < 0.01$). These findings suggest that while cooperative guilds may excel in promoting member well-being and contributing to broader societal goals, they may struggle to generate sufficient profits and compete in a market-driven environment.

Hybrid structures were positively associated with adaptability ($\beta = 0.34$, $p < 0.01$) and stakeholder engagement ($\beta = 0.29$, $p < 0.01$), but negatively associated with efficiency ($\beta = -0.18$, $p < 0.05$) and member identification ($\beta = -0.23$, $p < 0.01$). These results indicate that while hybrid guilds may be more effective in responding to diverse demands and building external relationships, they may face challenges in optimizing resource use and fostering a strong sense of belonging among members.

\section{Discussion}
\subsection{Implications for Organizational Theory and Practice}
The findings of this study have several implications for organizational theory and practice in the context of hunter guilds and beyond. First, they highlight the importance of contingency thinking in organizational design, suggesting that there is no one-size-fits-all structure that works best for all guilds. Instead, guilds must carefully consider their specific goals, challenges, and resources in choosing and adapting their structures over time.

Second, the findings underscore the trade-offs and tensions inherent in different organizational structures, such as the balance between efficiency and flexibility, or between control and autonomy. Guilds must be mindful of these trade-offs and seek to manage them through ongoing communication, negotiation, and adjustment.

Third, the findings point to the value of hybrid and pluralistic approaches to organizing, which combine elements of different structures to leverage their respective strengths and mitigate their weaknesses. Guilds that are able to integrate hierarchical, networked, and cooperative principles in a coherent and flexible manner may be best positioned to thrive in the complex and dynamic environment of the post-catastrophe world.

Fourth, the findings suggest that organizational structure is not just a technical or instrumental issue, but also a social and symbolic one. The choice of structure reflects and shapes the values, norms, and identities of guild members, and can have profound implications for their motivation, well-being, and sense of purpose. Guilds must therefore pay attention to the human and cultural dimensions of structure, and strive to create arrangements that are not only effective but also meaningful and empowering for their members.

\subsection{Limitations and Future Research Directions}
This study has several limitations that should be acknowledged and addressed in future research. First, while the sample of 500 guilds is relatively large and diverse, it is still limited to the Japanese context and may not be fully representative of hunter guilds in other countries or regions. Future studies could expand the scope of the investigation to include guilds from different cultural, political, and economic settings, and explore how these contextual factors influence the adoption and performance of different organizational structures.

Second, the cross-sectional nature of the study does not allow for a conclusive assessment of the causal relationships between organizational structure and performance outcomes. Future research could employ longitudinal or experimental designs to better establish the directionality and magnitude of these effects, and to examine how structures evolve and adapt over time in response to changing circumstances.

Third, while the study considered a range of performance outcomes, it did not fully capture the complexity and multidimensionality of guild effectiveness. Future studies could develop more comprehensive and nuanced measures of performance that take into account the diverse goals, stakeholders, and impacts of hunter guilds, such as their contributions to scientific knowledge, environmental sustainability, and social equity.

Fourth, the study relied primarily on self-reported data from guild leaders and members, which may be subject to biases and limitations. Future research could triangulate these subjective accounts with more objective and independent sources of data, such as archival records, external assessments, and behavioral observations.

Finally, the study focused on the formal and structural aspects of hunter guilds, but did not fully explore the informal and dynamic processes that shape their functioning and performance, such as leadership, communication, conflict, and learning. Future research could adopt a more processual and practice-oriented perspective to examine how guild members interact, negotiate, and improvise within and across different structural arrangements, and how these micro-level dynamics aggregate to macro-level outcomes.

\section{Conclusion}
The emergence of portals and dungeons in the wake of the 2025 catastrophe has given rise to a new breed of organizations: hunter guilds. These guilds have become critical actors in the post-catastrophe world, shaping the economic, social, and political landscape through their exploration, resource extraction, and monster-fighting activities. Yet, little is known about how these guilds are structured and how their structures influence their performance and adaptability.

This study has sought to address this gap by investigating the organizational structures of hunter guilds in Japan. Through a mixed-methods approach combining a nationwide survey and in-depth interviews, the study has identified four main types of structures: hierarchical, networked, cooperative, and hybrid. Each of these structures has its own advantages and disadvantages, and is influenced by factors such as guild size, dungeon complexity, member specialization, and trust.

The study has also examined the impact of organizational structure on various performance outcomes, revealing trade-offs and tensions between efficiency, flexibility, member satisfaction, and social impact. These findings underscore the importance of contingency thinking, hybrid approaches, and attention to the human and cultural dimensions of organizing.

As the first comprehensive study of hunter guild structures, this research makes important contributions to organizational theory and practice. It extends existing theories to a novel and extreme context, highlights the challenges and opportunities of organizing in the face of radical uncertainty, and provides practical insights for guild leaders, policymakers, and stakeholders.

However, the study also has limitations and raises new questions for future research. As the post-catastrophe world continues to evolve and new challenges emerge, it will be crucial to deepen and broaden our understanding of how hunter guilds and other organizations can adapt and thrive in this unprecedented era of human history.

\section*{Acknowledgements}
The authors would like to thank the Japan Society for the Promotion of Science (JSPS) for funding this research through Grant-in-Aid for Scientific Research (A) (Grant Number: JP12345678). We also express our gratitude to the participating hunter guilds for their time and insights, and to our research assistants for their diligent work in data collection and analysis.

\bibliographystyle{alpha}
\bibliography{references}

\end{document}
